
\documentclass{standalone}
\usepackage[UTF8,punct]{ctex}

\usepackage{graphicx}
\usepackage{color}
\usepackage{xcolor}
\definecolor{darkgreen}{rgb}{0.13,0.53,0.53}

\usepackage{amsfonts, amsmath} 
\usepackage[condensed, math]{iwona}

\usepackage{ifthen}
\usepackage{bm}

\usepackage{pgfplots}
\usepackage{pgfplotstable}
\usepackage{pgfcalendar}
\pgfplotsset{compat=newest}
\usepgfplotslibrary{groupplots}
\usepgfplotslibrary{fillbetween}

\usepackage{tikz}
\usetikzlibrary{
  arrows,
  automata,
  backgrounds,
  calc,
  calendar,
  chains,
  decorations,
  decorations.pathreplacing,
  decorations.pathmorphing,
  decorations.markings,
  decorations.text,
  er,
  external,
  fadings,
  fit,
  fixedpointarithmetic,
  fpu,
  folding,
  intersections,
  matrix,
  mindmap,
  lindenmayersystems,
  positioning,
  patterns,
  petri,
  plotmarks,
  shapes,
  shadows,
  shadings,
  snakes,
  spy,
  topaths,
  trees,
  turtle,
  through}


\begin{document}

\tikzset{
  every path/.style = {
    ->,
   draw = darkgreen,
   > = stealth, 
   thick,
   rounded corners}, % every path 是度所有的路径设置
  %   every node/.style = {
  %   rectangle,
  %   rounded corners,
  %   draw=cyan,
  %   fill=green!10,
  %   thick,
  %   minimum height=2em,
  %   inner sep=10pt,
  %   text centered,
  %   text width = 3cm,
  % }, every node 是对所有的节点设置。
  %   而下面这个state是一种参数设置,把它放到node后的[]才能生效
 % 而用every node的话就不用再在[]中设置
  state/.style = {
    rectangle,
    rounded corners,
    %draw=cyan,
    fill= gray!10,
    thin,
    minimum height=2em,
    inner sep=10pt,
    text centered,
    text width = 6cm,
  },
  note/.style = {
    fill = darkgreen!10,
    text width = 3cm,
    text centered,
  },
}

% We need layers to draw the block diagram
\pgfdeclarelayer{background}
\pgfdeclarelayer{foreground}
\pgfsetlayers{background,main,foreground}

\kaishu
  \begin{tikzpicture}
    
  \node[state] (kongjian)  {金融市场风险的演化空间};
  \node[state,below = 1cm of kongjian] (fenbu) {金融市场服从的无穷维的概率分布};  
  \node[state,below = 1cm of fenbu] (leixing) {概率分布的类型与参数};
  \node[note,left = 1cm of leixing] (unknown) {类型与参数的信息不完全};
  \node[state,below = 1cm of leixing] (sifi) {系统性金融风险的性质};

  
  
  \draw[<->,dashed]       (kongjian) to node[midway,left]  {概率论视角} (fenbu) ;
  
  \draw[->,dashed] (fenbu) to node [midway,left] {决定} (leixing);
  
  \draw[->,dashed] (leixing) to node [midway,left] (nopass) {反映} (sifi);
  \node[note,right = 4cm of nopass] (noinfor) {不完全信息使这条路径无法实现};
  
  \draw[->,dashed] (noinfor) to (nopass);
  \draw[->,dashed] (unknown) to (leixing);

 
  
\end{tikzpicture}


\end{document}