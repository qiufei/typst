
\documentclass{standalone}
\usepackage[UTF8,punct]{ctex}

\usepackage{graphicx}
\usepackage{color}
\usepackage{xcolor}
\definecolor{darkgreen}{rgb}{0.13,0.53,0.53}

\usepackage{amsfonts, amsmath} 
\usepackage[condensed, math]{iwona}

\usepackage{ifthen}
\usepackage{bm}

\usepackage{pgfplots}
\usepackage{pgfplotstable}
\usepackage{pgfcalendar}
\pgfplotsset{compat=newest}
\usepgfplotslibrary{groupplots}
\usepgfplotslibrary{fillbetween}

\usepackage{tikz}
\usetikzlibrary{
  arrows,
  automata,
  backgrounds,
  calc,
  calendar,
  chains,
  decorations,
  decorations.pathreplacing,
  decorations.pathmorphing,
  decorations.markings,
  decorations.text,
  er,
  external,
  fadings,
  fit,
  fixedpointarithmetic,
  fpu,
  folding,
  intersections,
  matrix,
  mindmap,
  lindenmayersystems,
  positioning,
  patterns,
  petri,
  plotmarks,
  shapes,
  shadows,
  shadings,
  snakes,
  spy,
  topaths,
  trees,
  turtle,
  through}


\begin{document}

\tikzset{
  every path/.style = {
    ->,
   draw = darkgreen,
   > = stealth, 
   thick,
   rounded corners}, % every path 是度所有的路径设置
  %   every node/.style = {
  %   rectangle,
  %   rounded corners,
  %   draw=cyan,
  %   fill=green!10,
  %   thick,
  %   minimum height=2em,
  %   inner sep=10pt,
  %   text centered,
  %   text width = 3cm,
  % }, every node 是对所有的节点设置。
  %   而下面这个black是一种参数设置,把它放到node后的[]才能生效
 % 而用every node的话就不用再在[]中设置
  black/.style = {
    rectangle,
    rounded corners,
    draw= black, % 这行是增加node的边框线
    %fill=gray!5,
    ultra thin,
    minimum height=1em,
    inner sep=10pt,
    text centered,
    text width = 6cm,
  },
  green/.style = {
    rectangle,
    rounded corners,
    draw= black, % 这行是增加node的边框线
    fill=darkgreen!10,
    ultra thin,
    minimum height=1em,
    inner sep=10pt,
    text centered,
    text width = 6cm,
  },
  blue/.style = {
    rectangle,
    rounded corners,
    draw= black, % 这行是增加node的边框线
    fill= blue!30,
    ultra thin,
    minimum height=1em,
    inner sep=10pt,
    text centered,
    text width = 6cm,
  }
  % grow=right,% 控制child的生长方向
  % level 1/.style={sibling distance=1.5cm},
  % level 2/.style={sibling distance=1cm},
  % level distance=6cm,
  %edge from parent fork right,
}

% We need layers to draw the block diagram
\pgfdeclarelayer{background}
\pgfdeclarelayer{foreground}
\pgfsetlayers{background,main,foreground}

\kaishu
  \begin{tikzpicture}

    \node[black] (kongjian) {将系统性金融风险的演化空间分解为风险能力分布和风险负担分布};
    \node[blue,left = of kongjian] (objective) {金融市场结构的客观部分};
    
    
    \node[black,below = of kongjian] (bm)  {以市场的行为鞅结构来进一步简化系统性金融风险的演化空间};
    \node[green,left =  of bm] (subjective) {金融市场结构的主观部分};
    \node[green,right = of bm] (game) {建立风险排序博弈 \\ 分析行为鞅的微观经济基础};


    
    
    \node[black,below = of bm] (cedu) {以坎托罗维奇对偶建立基于市场结构的系统性金融风险测度};
    \node[green,right = of cedu] (op)  {最优输运法};

    \node[black,below = of cedu] (r) {对不同行为鞅结构下的系统性金融风险进行数值模拟};
    \node[green,right = of r] (macro) {R语言,\\Gurobi软件};

  
  
    \draw[->] (kongjian) to  (bm) ;
    \draw[->] (objective) to  (kongjian) ;
    
    \draw[->] (bm)  to  (cedu);
    \draw[->] (subjective) to  (bm) ;
    \draw[->] (game) to  (bm) ;
    
       
    \draw[->] (cedu) to  (r);
    \draw[->] (op) to  (cedu);
    
    \draw[->] (macro) to (r);

    

    \node[draw,dashed,fit=(objective)(subjective),inner sep=2ex] (box) {};
    
    \node[left = of box] (review) {一些绝妙的点子};
    \draw[->] (review) to (box);
    
  

  
  
\end{tikzpicture}


\end{document}