
\documentclass{standalone}
\usepackage[UTF8,punct]{ctex}

\usepackage{graphicx}
\usepackage{color}
\usepackage{xcolor}
\definecolor{darkgreen}{rgb}{0.13,0.53,0.53}

\usepackage{amsfonts, amsmath} 
\usepackage[condensed, math]{iwona}

\usepackage{ifthen}
\usepackage{bm}

\usepackage{pgfplots}
\usepackage{pgfplotstable}
\usepackage{pgfcalendar}
\pgfplotsset{compat=newest}
\usepgfplotslibrary{groupplots}
\usepgfplotslibrary{fillbetween}

\usepackage{tikz}
\usetikzlibrary{
  arrows,
  automata,
  backgrounds,
  calc,
  calendar,
  chains,
  decorations,
  decorations.pathreplacing,
  decorations.pathmorphing,
  decorations.markings,
  decorations.text,
  er,
  external,
  fadings,
  fit,
  fixedpointarithmetic,
  fpu,
  folding,
  intersections,
  matrix,
  mindmap,
  lindenmayersystems,
  positioning,
  patterns,
  petri,
  plotmarks,
  shapes,
  shadows,
  shadings,
  snakes,
  spy,
  topaths,
  trees,
  turtle,
  through}


\begin{document}

\tikzset{
  every path/.style = {
    ->,
   draw = darkgreen,
   > = stealth, 
   thick,
   rounded corners}, % every path 是度所有的路径设置
  %   every node/.style = {
  %   rectangle,
  %   rounded corners,
  %   draw=cyan,
  %   fill=green!10,
  %   thick,
  %   minimum height=2em,
  %   inner sep=10pt,
  %   text centered,
  %   text width = 3cm,
  % }, every node 是对所有的节点设置。
  %   而下面这个state是一种参数设置,把它放到node后的[]才能生效
 % 而用every node的话就不用再在[]中设置
  state/.style = {
    rectangle,
    rounded corners,
    %draw=cyan,
    fill=gray!10,
    thin,
    minimum height=2em,
    inner sep=10pt,
    text centered,
    text width = 6cm,
  },
  note/.style = {
    fill = darkgreen!10,
    text width = 3cm,
    text centered,
  },
}

% We need layers to draw the block diagram
\pgfdeclarelayer{background}
\pgfdeclarelayer{foreground}
\pgfsetlayers{background,main,foreground}

\kaishu
  \begin{tikzpicture}
    
  \node[state] (kongjian)  {金融市场风险的演化空间};
  \node[state,below = 1cm of kongjian] (fenbu) {系统性金融风险服从的概率分布空间};
  \node[state,below = 1cm of fenbu] (jiegou)  {金融市场的结构}; 
  \node[state,below = 1cm of jiegou] (junheng) {金融市场的均衡类型};
  \node[state,below = 1cm of junheng] (yang) {行为鞅的结构};
  \node[state,below = 1cm of yang] (sifi) {系统性金融风险的性质};

  
  
  \draw[<->] (kongjian) to node[right] {概率论视角} (fenbu) ;
  \draw[->] (fenbu)    to node [midway,right]   {决定} (jiegou);
  \draw[<->] (jiegou)  to node [midway,right] {影响} (junheng);
  \draw[->,transform canvas={xshift= 9pt}]  (junheng) to node [midway,right]   {决定}     (yang);
  \draw[->,transform canvas={xshift=-9pt}]  (yang)     to node [midway,left]   {描述} (junheng);
  \draw[->] (yang) -- node [midway,right] {反映} (sifi);


  \path[]  (jiegou)  edge [->,bend left = 95] node[note,left] {用结构限定\\概率分布空间}  (fenbu);

  \path[]  (yang)  edge [->,bend right = 95] node[note,right] {行为鞅赋予结构\\经济行为含义}  (jiegou);
  
  
\end{tikzpicture}


\end{document}