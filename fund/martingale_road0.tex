
\documentclass{standalone}
\usepackage[UTF8,punct]{ctex}

\usepackage{graphicx}
\usepackage{color}
\usepackage{xcolor}
\definecolor{darkgreen}{rgb}{0.13,0.53,0.53}

\usepackage{amsfonts, amsmath} 
\usepackage[condensed, math]{iwona}

\usepackage{ifthen}
\usepackage{bm}

\usepackage{pgfplots}
\usepackage{pgfplotstable}
\usepackage{pgfcalendar}
\pgfplotsset{compat=newest}
\usepgfplotslibrary{groupplots}
\usepgfplotslibrary{fillbetween}

\usepackage{tikz}
\usetikzlibrary{
  arrows,
  automata,
  backgrounds,
  calc,
  calendar,
  chains,
  decorations,
  decorations.pathreplacing,
  decorations.pathmorphing,
  decorations.markings,
  decorations.text,
  er,
  external,
  fadings,
  fit,
  fixedpointarithmetic,
  fpu,
  folding,
  intersections,
  matrix,
  mindmap,
  lindenmayersystems,
  positioning,
  patterns,
  petri,
  plotmarks,
  shapes,
  shadows,
  shadings,
  snakes,
  spy,
  topaths,
  trees,
  turtle,
  through}


\begin{document}

\tikzset{
    vertex/.style = {
        circle,
        fill            = black,
        outer sep = 2pt,
        inner sep = 1pt,
      }   
  }

\tikzset{
    force/.style = {
      text width = 2cm,
      align = center
      }   
    }

\tikzset{
  every path/.style = {
    ->,
   draw = darkgreen,
   > = stealth, 
   very thick,
   rounded corners}, % every path 是度所有的路径设置
  %   every node/.style = {
  %   rectangle,
  %   rounded corners,
  %   draw=cyan,
  %   fill=green!10,
  %   thick,
  %   minimum height=2em,
  %   inner sep=10pt,
  %   text centered,
  %   text width = 3cm,
  % }, every node 是对所有的节点设置。
  %   而下面这个state是一种参数设置,把它放到node后的[]才能生效
 % 而用every node的话就不用再在[]中设置
  state/.style = {
    rectangle,
    rounded corners,
    draw=cyan,
    fill=gray!10,
    thick,
    minimum height=2em,
    inner sep=10pt,
    text centered,
    text width = 3cm,
  },
  note/.style = {
    fill = green!10,
    text width = 1cm,
    text centered,
  },
  %node distance=2cm and 1cm,
}
    
  
  \begin{tikzpicture}
    
  \node[state] (kongjian)    at (7, 0) {金融市场风险的演化空间};
  \node[state] (fenbu)       at (7,-3) {无限维的市场整体的分布};
  \node[state,dotted,text width =1.5cm] (leixing)     at (2.5,-6) {分布的类型与参数};
  
  \node[state] (sifi)       at (7,-12) {系统性金融风险的性质};

  \node[note] (buwanquan) at (0,-6) {不完全信息使得这些数据无法获得};
  
  \draw (kongjian) to (fenbu);
  \draw[dashed,->] (fenbu) -| node [near start,above] {决定} (leixing);
  \draw[dashed,->] (leixing) |- node [near start,below,midway,below] {反映} (sifi);
  \draw[dashed,->] (buwanquan) to (leixing); 
  

  

  \node[state] (junheng)  at (13,- 3) {金融市场均衡的类型};
  \node[state] (shichang) at (13,- 6) {金融市场的结构};
  \node[state] (yang)     at (7, - 6) {行为鞅(behavior martingale)的性质};
  
  
  \draw[->] (fenbu)    to node [midway,above]   {决定} (junheng);
  \draw[<->] (junheng)  to node [midway,right] {反映} (shichang);
  \draw[->,transform canvas={yshift= 9pt}]  (shichang) to node [midway,above]   {决定}     (yang);
  \draw[->,transform canvas={yshift=-9pt}]  (yang)     to node [midway,below]   {描述} (shichang);
  \draw[->] (yang) -- node [midway,right] {反映} (sifi);
  
  
\end{tikzpicture}


\end{document}