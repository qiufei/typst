
\documentclass{standalone}
\usepackage[UTF8,punct]{ctex}

\usepackage{graphicx}
\usepackage{color}
\usepackage{xcolor}
\definecolor{darkgreen}{rgb}{0.13,0.53,0.53}

\usepackage{amsfonts, amsmath} 
%\usepackage[condensed, math]{iwona}

\usepackage{ifthen}
\usepackage{bm}

\usepackage{pgfplots}
\usepackage{pgfplotstable}
\usepackage{pgfcalendar}
\pgfplotsset{compat=newest}
\usepgfplotslibrary{groupplots}
\usepgfplotslibrary{fillbetween}

\usepackage{tikz}
\usetikzlibrary{
  arrows,
  automata,
  backgrounds,
  calc,
  calendar,
  chains,
  decorations,
  decorations.pathreplacing,
  decorations.pathmorphing,
  decorations.markings,
  decorations.text,
  er,
  external,
  fadings,
  fit,
  fixedpointarithmetic,
  fpu,
  folding,
  intersections,
  matrix,
  mindmap,
  lindenmayersystems,
  positioning,
  patterns,
  petri,
  plotmarks,
  shapes,
  shadows,
  shadings,
  snakes,
  spy,
  topaths,
  trees,
  turtle,
  through}

%%%%%%%%%%%%%%%%%%%%%%%%


\begin{document}

% 系统性金融风险的演化空间:结构确定

\kaishu
\begin{tikzpicture}
  [domain=0:10, xscale=1, yscale=7,
   aponta/.style={
    color=darkgreen!80, rounded corners=5pt, -latex, thick
  }]
  
  \draw[-latex] (0 , 0) --  (10  , 0)  node[right]          {时间轴($T$)};
  \draw[-latex] (0 , 0) --  (0  ,  1)   node[left,align =center]  (E)     {市场结构($S$): \\ 行为鞅};

  \def\A{-3}; \def\B{0.8};
  
  \draw[color=black, thick,domain=0:9]
  plot[id=x] function{1/(1+exp(-\A-\B*x))}
  % node[right,align =center,text width =4.5cm] (eta) % 在上面画出来的图像的终点的右边,建立一个节点,节点名字叫eta,内容是下面的表达式
  % {金融风险演化空间的表述:\\ 一系列具体的分布};

  node[align =center] (eta)  at (11,1)% 在上面画出来的图像的终点的右边,建立一个节点,节点名字叫eta,内容是下面的表达式
  {风险演化空间的表述:\\ 一系列具体的分布};
  
  \node (Q) at (5,27/20) {市场结构决定系统性金融风险的性质};
  
  \node[align =center] (N) at (5,22/20) {某一时间和行为鞅组合$(t_{i},s_{i})$下 \\ 系统性金融风险的分布};
  
  \path[aponta] (N) edge[bend right=10]  (3.9,12/20);
  \draw[aponta] (E) |- (Q);
  \draw[aponta] (Q) -| (eta);
  
  \def\parphi{40}; 
  \def\parmu{0.5};
  \def\factor{0.14};
  
  \foreach[evaluate=\x as \parmu using 1/(1+exp(-\A-\B*\x))]
    \x in {1, 2.4, ..., 7}{
      \draw[color=gray, dashed] (\x,0) -- ++(0,1);
      \begin{scope}[xshift=\x cm, rotate=90, yscale=1]
        % \draw (\parmu,0) -- ++(0,-0.5);
        \filldraw[fill opacity=0.3, fill=darkgreen!35]
          plot[domain=(\parmu-0.25):(\parmu+0.25), smooth, samples=100]
          function {-\factor*gamma(\parphi)/(gamma(\parphi*\parmu)*gamma((1-\parmu)*\parphi))*x**(\parmu*\parphi-1)*(1-x)**((1-\parmu)*\parphi-1)};
      \end{scope}
    }
    

\end{tikzpicture}

\end{document}