\documentclass{standalone}
\usepackage[UTF8,punct]{ctex}

\usepackage{graphicx}
\usepackage{color}
\usepackage{xcolor}
\definecolor{darkgreen}{rgb}{0.13,0.53,0.53}
\definecolor{blue}{RGB}{173, 216, 230}
\usepackage{amsfonts, amsmath} 
\usepackage[condensed, math]{iwona}

\usepackage{ifthen}
\usepackage{bm}

\usepackage{pgfplots}
\usepackage{pgfplotstable}
\usepackage{pgfcalendar}
\pgfplotsset{compat=newest}
\usepgfplotslibrary{groupplots}
\usepgfplotslibrary{fillbetween}

\usepackage{tikz}
\usetikzlibrary{
  arrows,
  automata,
  backgrounds,
  calc,
  calendar,
  chains,
  decorations,
  decorations.pathreplacing,
  decorations.pathmorphing,
  decorations.markings,
  decorations.text,
  er,
  external,
  fadings,
  fit,
  fixedpointarithmetic,
  fpu,
  folding,
  intersections,
  matrix,
  mindmap,
  lindenmayersystems,
  positioning,
  patterns,
  petri,
  plotmarks,
  shapes,
  shadows,
  shadings,
  snakes,
  spy,
  topaths,
  trees,
  turtle,
  through}


\begin{document}

\tikzset{
  every path/.style = {
    ->,
   draw = black,
   > = stealth, 
   thick,
   rounded corners}, % every path 是对所有的路径设置
  %   every node/.style = {
  %   rectangle,
  %   rounded corners,
  %   draw=cyan,
  %   fill=green!10,
  %   thick,
  %   minimum height=2em,
  %   inner sep=10pt,
  %   text centered,
  %   text width = 3cm,
  % }, every node 是对所有的节点设置。
  %   而下面这个black是一种参数设置,把它放到node后的[]才能生效
 % 而用every node的话就不用再在[]中设置
  black/.style = {
    rectangle,
    rounded corners,
    draw= black, % 这行是增加node的边框线
    ultra thin,
    minimum height=1em,
    inner sep=10pt,
    text centered,
    text width = 6cm,
  },
  green/.style = {
    rectangle,
    rounded corners,
    draw= black, % 这行是增加node的边框线
    fill=darkgreen!10,
    ultra thin,
    minimum height=1em,
    inner sep=10pt,
    text centered,
    text width = 6cm,
  },
  blue/.style = {
    rectangle,
    rounded corners,
    draw= black, % 这行是增加node的边框线
    fill= blue,
    ultra thin,
    minimum height=1em,
    inner sep=10pt,
    text centered,
    text width = 6cm,
  }
}

% We need layers to draw the block diagram
\pgfdeclarelayer{background}
\pgfdeclarelayer{foreground}
\pgfsetlayers{background,main,foreground}

\kaishu
  \begin{tikzpicture}

%%% 绪论

    \node[blue] (intro) {绪论};
    \node[black,below left = of intro] (intro_left) {高货币化的社会背景};
    \node[green,below = of intro] (intro_middle) {风险后果成为婚姻决策的重要决定因素};
    \node[black,below right = of intro] (intro_right) {金融风险的个体感知};

    \draw[->] (intro) -- (intro_middle);
    % \draw[->]  (intro.south) -- (intro_left);
    % \draw[->]  (intro.south) -- (intro_right);

    \draw[->]  ($(intro)!0.5!(intro_middle)$) -| (intro_left);

    \draw[->]  ($(intro)!0.5!(intro_middle)$) -| (intro_right);


    \node[draw,dashed,darkgreen,fit=(intro)(intro_left)(intro_middle)(intro_right),inner sep=2ex] (intro_box) {};
    
    \node[left = of intro_box] (intro_note) {提出问题};

%%% 文献梳理

    \node[blue, below = of intro_box] (literature) {理论基础};

    \node[black,below left = of literature] (literature_left) {微观婚姻经济学};
    \node[green,below = of literature] (literature_middle) {金融风险因素研究的不足};
    \node[black,below right = of literature] (literature_right) {宏观婚姻经济学};

    \draw[->] (literature) -- (literature_middle);

    \draw[->]  ($(literature)!0.5!(literature_middle)$) -| (literature_left);

    \draw[->]  ($(literature)!0.5!(literature_middle)$) -| (literature_right);


    \node[draw,dashed,darkgreen,fit=(literature)(literature_left)(literature_middle)(literature_right),inner sep=2ex] (literature_box) {};
    
    \node[left = of literature_box] (literature_note) {文献梳理};


%%%% 机制分析
    \node[blue, below = of literature_box] (mechanism) {机制分析};

    \node[black,below left =  of mechanism]  (mechanism_left)   {高货币化对风险环境的影响机制};
    \node[green,below = of mechanism]        (mechanism_middle) {核心假设:\\ 金融风险分担效果影响婚姻决策 };
    \node[black,below right = of mechanism]  (mechanism_right)  {金融风险对结婚率的影响机制};

    \draw[->] (mechanism) -- (mechanism_middle);
    \draw[->]  ($(mechanism)!0.5!(mechanism_middle)$) -| (mechanism_left);
    \draw[->]  ($(mechanism)!0.5!(mechanism_middle)$) -| (mechanism_right);


% 高货币化机制
    \node[black,below = of mechanism_left]  (money_1)   {金融风险影响个体财富安全};
    \node[black,below = of money_1]         (money_2) {经济压力与结婚决策延迟};
    \node[black,below = of money_2]         (money_3)  {就业市场稳定性受冲击};

    \draw[->] (mechanism_left) -- (money_1);
    \draw[->] (money_1) -- (money_2);
    \draw[->] (money_2) -- (money_3);



% 核心假设
    \node[green,below = of mechanism_middle]  (assumption_1)  {风险分担效应};
    \node[green,below = of assumption_1]      (assumption_2)  {谨慎选择效应};

    % \draw (mechanism_middle) -- (assumption_1);

    % \draw (assumption_1) -- (assumption_2);

    % 机制分析框    
    \node[draw,dashed,fit=(mechanism_middle)(assumption_1)(assumption_2),inner sep=1ex] (main_assumption) {};

    \node[below = of main_assumption] (assumption_note) {};


% 风险传导机制

    \node[black,below = of mechanism_right] (risk_1) {金融风险与婚姻收益-成本分析 };
    \node[black,below = of risk_1]          (risk_2) {金融风险对家庭规模与结构的影响};
    \node[black,below = of risk_2]          (risk_3) {金融风险与婚姻市场的供求关系};
    \node[black,below = of risk_3]          (risk_4) {金融风险与婚姻满意度及稳定性};

    \draw[->] (mechanism_right) -- (risk_1);
    \draw[->] (risk_1) -- (risk_2);
    \draw[->] (risk_2) -- (risk_3);
    \draw[->] (risk_3) -- (risk_4);



% 机制分析框    
    \node[draw,dashed,darkgreen,fit=(mechanism)(mechanism_left)(mechanism_middle)(mechanism_right)(money_1)(money_2)(money_3)(assumption_1)(assumption_2)(risk_1)(risk_2)(risk_3)(risk_4),inner sep=2ex] (mechanism_box) {};

    \node[left = of mechanism_box] (mechanism_note) {分析问题};


%%%% 实证检验
    
    \node[blue, below = of mechanism_box] (positive) {实证分析};

    \node[black,below left = of positive]  (positive_1) {基准分析:DID模型 };
    \node[black,below = of positive]       (positive_2) {动态分析:数值模拟};
    \node[black,below right = of positive] (positive_3) {异质性分析:收入、地区、教育等};

    \draw[->] (positive) -- (positive_2);
    \draw[->]  ($(positive)!0.5!(positive_2)$) -| (positive_1);
    \draw[->]  ($(positive)!0.5!(positive_2)$) -| (positive_3);


% 实证分析框    
    \node[draw,dashed,darkgreen,fit=(positive)(positive_1)(positive_2)(positive_3),inner sep=2ex] (positive_box) {};

    \node[left = of positive_box] (positive_note) {实证检验};


%%%% 政策建议

    \node[blue, below = of positive_box] (policy) {政策建议};

    \node[black,below left = of policy]  (policy_1) {针对金融风险的政策调控建议};
    \node[black,below right = of policy] (policy_2) {针对婚姻决策的政策引导建议};

    \draw[->] (policy) -- (policy_1);
    \draw[->] (policy) -- (policy_2);

% 政策建议框
    \node[draw,dashed,darkgreen,fit=(policy)(policy_1)(policy_2),inner sep=2ex] (policy_box) {};

    \node[left = of policy_box] (policy_note) {政策建议};

%%%% 结论
  
      \node[blue, below = of policy_box] (conclusion) {结论};

      \node[black,below left  = of conclusion]  (conclusion_1) {主要结论};
      \node[green,below       = of conclusion]  (conclusion_2) {创新点};
      \node[black,below right = of conclusion]  (conclusion_3) {研究展望};
      
      \draw[->] (conclusion) -- (conclusion_2);
      \draw[->]  ($(conclusion)!0.5!(conclusion_2)$) -| (conclusion_1);
      \draw[->]  ($(conclusion)!0.5!(conclusion_2)$) -| (conclusion_3);
      
  
% 结论框
      \node[draw,dashed,darkgreen,fit=(conclusion)(conclusion_1)(conclusion_2)(conclusion_3),inner sep=2ex] (conclusion_box) {};
  
      \node[left = of conclusion_box] (conclusion_note) {研究总结};



%%%% 全图纵贯线

    \draw[->,darkgreen,line width=3pt] (intro_box) -- (literature_box);
  \end{tikzpicture}


\end{document}