
\documentclass[a4paper]{article}
\usepackage[UTF8,punct]{ctex}

\usepackage[]{graphicx}
\usepackage[]{color}

\usepackage[brazil]{babel}
\usepackage[utf8]{inputenc}

% \usepackage{palatino}
% \usepackage{eulervm}
\usepackage[condensed, math]{iwona}
% \usepackage[scaled=0.8]{beramono}
\usepackage{inconsolata}


\usepackage[T1]{fontenc}


\usepackage{scrextend}
% \changefontsizes[11pt]{8pt}
\usepackage{array}
\usepackage{colortbl}
\usepackage{booktabs}
\usepackage{filecontents}
\usepackage{multirow}
\usepackage{eurosym}
\usepackage{amsfonts, amssymb, amsxtra, amsmath, amsbsy} 

\usepackage{xcolor}
\usepackage{listings}
\usepackage{ifthen}

\usepackage{tikz,bm}
\usepackage{pgfplots}

\usepackage{pgfplotstable}
\usepackage{pgfcalendar}
\pgfplotsset{compat=newest}
\usepgfplotslibrary{groupplots}
\usepgfplotslibrary{fillbetween} 
\usetikzlibrary{
  arrows,
  through,
  positioning,
  matrix,
  calc,
  decorations.pathreplacing,
  decorations.pathmorphing,
  decorations.markings,
  decorations.text,
  shapes,
  backgrounds,
  shadows,
  trees,
  fit,
  fadings,
  snakes,
  patterns,
  mindmap,
  intersections,
  calendar,
  plotmarks,
  spy}


\pagestyle{empty}


\definecolor{darkgreen}{rgb}{0.13,0.53,0.53}




\begin{document}


\section{系统性金融风险的分布}

\pgfplotsset{width=7cm,compat=1.8}
\pgfplotsset{%
  colormap={whitered}{color(0cm)=(white);
    color(1cm)=(orange!75!red)}
}

\begin{tikzpicture}[
  declare function={mu1=1;},
  declare function={mu2=2;},
  declare function={sigma1=0.45;},
  declare function={sigma2=0.6;},
  declare function={
    normal(\m,\s)=1/(2*\s*sqrt(pi))*exp(-(x-\m)^2/(2*\s^2));
  },
  declare function={
    bivar(\ma,\sa,\mb,\sb)=
    1/(2*pi*\sa*\sb)*exp(-((x-\ma)^2/\sa^2+(y-\mb)^2/\sb^2))/2;
  }
]

  \begin{axis}[
    colormap name=whitered,
    width=11cm,
    view={45}{45},
    enlargelimits=false,
    grid=major,
    domain=0:2,
    y domain=0.5:3.5,
    samples=26,
    xlabel=$x$,
    ylabel=$y$,    
    zlabel={$P$},
    xticklabels=\empty,
    yticklabels=\empty,
    zticklabels=\empty,
    axis line style={draw=none},
    tick style={draw=none},
    scale = 1.5,
    legend style={at={(0.45,-0.08)},anchor=south,draw=none} % draw=none不要legend的外边框
    ]
    %   x,y的联合分布
    \addplot3 [surf] {bivar(mu1, sigma1, mu2, sigma2)};
    
    % x的边缘分布
    \addplot3 [domain=0:2, samples=31, samples y=0, thick, smooth,color = pink!70!red]
    (x, 3.5, {normal(mu1, sigma1)})
    node[midway,below,yshift = -2cm,text width = 2.5cm,align = center] {风险负担分布:\\ 边缘概率分布};
    % 这里第二个参数即y=3,是因为y值的范围终点是3
    
    
    % y的边缘分布
    \addplot3 [domain=0.5:3.5, samples=31, samples y=0, thick, smooth,color = darkgreen]
    (0, x, {normal(mu2, sigma2)})
    node[midway,below,yshift = -2cm,text width = 2.5cm,align =center] {风险能力分布:\\ 边缘概率分布};
    % 这里第一个参数为常数,代表在x=c的面上画图,这里c=0是因为前面设定x值的范围是从0开始,domain=0:2
    % 第二个参数变动,就是y值的范围,前面设定的y的范围是1:3
    
    \legend{系统性金融风险的联合概率分布}


    
    % 
    % \node[above] at (axis cs:0,1.5,0.28)  [pin=-15:风险能力分布$P(x_1)$] {边缘分布};
    % y 边际
    % \node at (axis cs:0.7,3,0.32) [pin=-15:风险负担分布$P(x_2)$] {};

    
  \end{axis}
\end{tikzpicture}


\section{系统性风险演化空间:结构未定}
\label{sec:test}

\begin{tikzpicture}[
  domain=0:10, xscale=0.8, yscale=0.4,
  aponta/.style={
    color=darkgreen!50, rounded corners=5pt, -latex, thick}
  ]
  \draw[-latex] (0,0) -- (12.5,0) node[right]     {时间轴 $T$};
  \draw[-latex] (0,0) -- (0,20) node[left]  (E) {市场结构 $S$};
  
  \def\A{19}; \def\V{2};
  \draw[color=black, thick, domain=0:11]
    plot[id=x] function{\A*x/(\V+x)}
    node[right] (eta) {}; %演化轨迹$1$
    
  \node (Q) at (5,25) {市场结构决定系统性金融风险的性质};
  \node (N) at (5,22) {某一具体时间点$t_{i}$上可能的风险分布族};
  
  \path[aponta]  (N) edge[bend right=10] (2.35,12);
 
  
  \draw[color=black, very thin, domain=0:11]
    plot[id=x, samples=100] function{(\A-5.5)*x/(\V+x)}
    node[right] {}; % 演化轨迹$2$
  \foreach \Aa in {-5.5,-2.7,-0.5,0,1.2,1.8,4.5}{
    \draw[color=black, very thin]
      plot[id=x, samples=100] function{(\A+\Aa)*x/(\V+x)};
    \foreach[evaluate=\x as \y using (\A+\Aa)*\x/(\V+\x)]
      \x in {0.5, 2.2, ..., 8.5}{
        \draw[color=gray, dashed] (\x,0) -- ++(0,20);
        \def\sr{2.5}
        \begin{scope}[
          xshift=\x cm, yshift=\y cm,
          rotate=-90, smooth, domain=-\sr:\sr]
          \filldraw[fill opacity=0.3, fill=white!85!black]
            plot[id=x] function{0.5*exp(-(x)**2/0.5)};
      \end{scope}
    }
  }
  
  \def\sr{8}
  \begin{scope}[
    xshift=9.1 cm, yshift=0.83*\A cm,
    rotate=-90, smooth, domain=-\sr:\sr]
    \filldraw[fill opacity=0.3, fill=gray!50!black]
      plot[id=x] function{1.5*exp(-(x)**2/10)};
    \node[right,text width = 3.5cm] (a) at (0,2) {演化空间整体的分布:未知的无穷维分布}; %
    %\draw[aponta] (1,0.5) |- (a);
  \end{scope}

 \draw[aponta] (E) |- (Q);
 \draw[aponta] (Q) -| (a);
  
  
\end{tikzpicture}



\section{最优输运,两个分布,不同颜色}

\begin{tikzpicture}[domain=-9:9,xscale=0.85, yscale=9,>=latex]

\draw[->] (-9,0) -- (7,0);  % node[right] {$x$};
% \draw[->] (0,0) -- (0,0.5) node[above] {$f(x)$};

\def\muone{-6}
\def\domainone{(\muone-2.5):(\muone+2.5)}
\def\sigma{0.7}

\def\mutwo{-2}
\def\domaintwo{(\mutwo-3.5):(\mutwo+3.5)}
\def\sigmatwo{1.1}

\node (c) at (-6,  -0.05)  {风险能力分布};

\node (ca) at (-6,0) {};


\node (r) at  (2, -0.4)  {风险负担分布};
\node (ri) at (2,0) {};
%%%%%%%%%%%%%%%%%%%%%%%%%%%%%%%%%%%%%%%%%%%%%%%%%%%%%%%%%%%%%%%%在node里可以加入xshift和yshift参数
\path[color=darkgreen!75!black, thick] (ca) edge[->,bend left = 5]  node[above] {最优输运过程,$\displaystyle\frac{\theta_a x}{\theta_v+x}$} (ri) ;

\draw[domain=\domainone,smooth,color = gray,fill= darkgreen!20]  plot (\x,{(2*3.1415*\sigma^2)^(-0.5)*exp(-(\x-\muone)^2/(2*\sigma^2))}) node[below] (capacity) {};


\draw[domain=\domaintwo,smooth,color = gray,fill = pink!25,rotate= -180]  plot (\x,{(2*3.1415*\sigmatwo^2)^(-0.5)*exp(-(\x-\mutwo)^2/(2*\sigmatwo^2))}) node[above] (risk)  {};

  
\end{tikzpicture}


\section{流程图加框加注释}

\tikzset{
  every path/.style = {
   ->,
   > = stealth, 
   rounded corners},
  state/.style = {
    fill = white,
    text centered
  },
  node distance=1.25cm,
  hlt/.style = {opacity = 0.7, line cap = round}
}%

\begin{tikzpicture}

  \node[state] (formular) {Formular};
  \node[state, right = 0.5cm of formular] (desenhar) {Desenhar};
  \node[state, below right of = desenhar] (coletar) {Coletar};
  \node[state, below right of = coletar] (armazenar) {Armazenar};
  \node[state, below right of = armazenar] (importar) {Importar};
  \node[state, above right of = importar] (manipular) {Arrumar};
  \node[state, above right of = manipular] (transformar) {Transformar};
  \node[state, above right = 0.3cm of transformar] (visualizar) {Visualizar};
  \node[state, below right = 0.3cm of transformar] (modelar) {Modelar};
  \node[state, right = 0.5cm of modelar] (comunicar) {Compreender};
  \node[state, below of = comunicar] (agir) {Agir};

  \path[draw] (formular) -- (desenhar);
  \path[draw] (desenhar) -- (coletar);
  \path[draw] (coletar) -- (armazenar);
  \path[draw] (armazenar) -- (importar);
  \path[draw] (importar) -- (manipular);
  \path[draw] (manipular) -- (transformar);

%   \path[draw] (transformar) edge[out=90, in=180] (visualizar);
%   \path[draw] (visualizar) edge[out=0, in=90] (modelar);
%   \path[draw] (modelar) edge[out=270, in=270] (transformar);

  \path[draw] (transformar) edge[bend left=30] (visualizar);
  \path[draw] (visualizar) edge[bend left=30] (modelar);
  \path[draw] (modelar) edge[bend left=30] (transformar);

  \path[draw] (modelar) -- (comunicar);
  \path[draw] (comunicar) -- (agir);

  \begin{pgfonlayer}{background}
  \node[hlt, 
    draw = blue,
%     fill = blue,
    fit = (desenhar)(importar),
    label = {[blue!70]below:Cientista da Computa{\c c}{\~a}o}] {};
  \node[hlt, 
    draw = red,
%     fill = red,
    fit = (importar)(visualizar),
    label = {[red!70]below:Estat{\'i}stico}] {};
  \node[hlt, 
    inner sep = 2em,
    draw = black, 
    fill = none,
    fit = (formular)(importar)(comunicar), 
    label = above:Cientista de Dados] {};
  \end{pgfonlayer}

\end{tikzpicture}%


\section{轨迹分布:教程}

\begin{tikzpicture}[domain=0:8, scale=0.7]
  
  % \draw[very thin,color=gray!30] (-1.1,-1.1) grid (12.1,7.1);
  
  \draw[->] (-0.2,0) -- (9.2,0) node[right] {$x$}; %画X轴,标签为X
  
  \draw[->] (0,-0.2) -- (0,5.5) node[above] {$E(Y|x)$}; %画Y轴,标签为E(Y|X)
  
  \draw[color=darkgreen, thick]
      plot[id=x] function{0+0.7*x} %画直线,斜率为0.7
      node[right] {$\mu = \beta_0+\beta_1\cdot x$};
      
% 在前面画的斜率为0.7的直线上的四个点(1.5,1.05),(3,2.1),(4.5,3.15),(6,4.2)画图
      % 第一个点的x坐标1.5是任意的,其纵坐标通过乘以直线的斜率0.7得到1.05
      % 后面三个点分别通过乘以第一个点坐标的2,3,4倍得到
  \foreach \x/\y in {1.5/1.05, 3/2.1, 4.5/3.15, 6/4.2}{
    \begin{scope}[
      xshift=\x cm,
      yshift=2cm +\y cm, % 这个平移的值跟下面纵向线的长度有关
      rotate=-90,% 顺时针转90度,
% 为什么要转,因为下面plot函数画的正态分布图像的开口默认是朝下的,现在我们想让它朝左
      smooth, domain=0.5:3.5]
      
      \draw[color=darkgreen] (1,0) -- (3,0); % 画图中过直线的纵向线
      % 本来是画从(1,0)到(3,0)的横线,
      % 但前面scope旋转了,所以以(1,0)为原点顺时针旋转90度
      % 于是直线变为从(1,-2)到(1,0)
      % 前面定义了坐标平移,平移过程是x坐标不变,y坐标加2
      % 于是直线在图中效果变为从(1,0)到(1,3)
      % 这里x坐标为什么是1呢?
      % 因为原始四个点中第一个点的坐标为(1.5,1.05),坐标起点是0
      % 但是在scope中,坐标起点是0.5,为了与之匹配,所以这里的x坐标变为1
      
      \draw[color=darkgreen!70!blue] (2,0) -- (2,1); % 画图中过直线的横向线
      % 详细变换同上
      % 这里为什么坐标从(2,0)的2开始呢?因为这个点要成为原始四个点的起点,
      % 原始起点的坐标是1.5,scope中domain的起点是0.5,所以这里变为2
      
      \filldraw[
        color=darkgreen!70!black, fill opacity=0.3, fill=darkgreen!70!black]
        plot[id=x] function{exp(-(x-2)**2/0.2)}; %分别以前面四个点为原点
    \end{scope}
  }
  
  \path[->, color=darkgreen!70!black, thick] (6.5,3.7) edge[bend right=45] 
    node[at end, right]
    {$\frac{1}{\sqrt{2\pi\sigma^2}}
      \exp\{-\frac{(y-\mu)^2}{2\sigma^2}\}$}
    +(1,-1); %这里的(1,-1)是调整前面那个标签距离图形的距离
    
\end{tikzpicture}


\begin{tikzpicture}[
  domain=0:9, xscale=0.8, yscale=0.25,
  aponta/.style={
    color=darkgreen!50!blue, rounded corners=5pt, -latex, thick
  }]

  \draw[-latex] (-0.2,0) -- (10,0) node[right] {$x$};
  \draw[-latex] (0,-0.2) -- (0,20) node[above] (E) {$\textrm{E}(Y|x)$};
  \def\A{18}; \def\V{2};
  \draw[color=black, thick]
    plot[id=x] function{\A*x/(\V+x)}
    node[right] (eta) {$\displaystyle\frac{\theta_a x}{\theta_v+x}$};
  \node (Q) at (5,25) {$Q(Y|x) = \eta(x, \theta)$};
  \node (N) at (5,22) {$[Y|x]\sim$ Normal($\mu$,$\sigma$)};
  \node[below right of=E, anchor=west] (V) {$\textrm{V}(Y|x)$};
  \path[aponta] (3.2,11.8) edge[bend left=10] (N);
  \draw[aponta] (Q) -| (E);
  \draw[aponta] (Q) -| (V);
  \draw[aponta] (Q) -| (eta);
  \foreach[
    evaluate=\x as \y using \A*\x/(\V+\x),
    evaluate=\y as \s using (\y^1.25)/20]
    \x in {0.8, 3, ..., 8}{
      \draw[color=gray, dashed] (\x,0) -- ++(0,20);
      \def\factor{2.3}
      \begin{scope}[
        xshift=\x cm, yshift=\y cm,
        rotate=-90, smooth, domain=(-4*\s):(4*\s)]
        \draw[color=gray] (-2*\s,0) -- (2*\s,0);
        \filldraw[fill opacity=0.3, fill=gray!70!black]
          plot[id=x]
          function{\factor*(2*3.14)**(-0.5)*(1/\s)*
            exp(-(x)**2/(2*\s**2))};
    \end{scope}
  }
\end{tikzpicture}





\section{回归公式图解}

\begin{tikzpicture}[
  mtx/.style={
    matrix of math nodes,
    left delimiter={[},
    right delimiter={]}
  },
  subtxt/.style={below, font=\footnotesize}]

  \matrix[mtx] (Y) {%
    y_1 \\ y_2 \\ \vdots \\ y_n\\
  };
  \matrix[mtx, right=of Y] (X) {%
    1      & x_{11} & \ldots & x_{1k} \\
    1      & x_{21} & \ldots & x_{2k} \\
    \vdots &        & \ddots & \vdots \\
    1      & x_{n1} & \ldots & x_{nk} \\
  };
  \matrix[mtx, right=0.5cm of X] (beta) {%
    \beta_0 \\ \beta_1 \\ \vdots \\ \beta_k\\
  };
  \matrix[mtx, right=of beta] (E) {%
    \epsilon_1 \\ \epsilon_2 \\ \vdots \\ \epsilon_n\\
  };

  \node at ($(Y.east)!0.5!(X.west)$) {$=$};%
  \node at ($(beta.east)!0.5!(E.west)$) {$=$};%
  \node[above=2ex of X] (modmat) {$y = X \beta+\epsilon$};
  \node[above=2ex of modmat] (mod)
    {$y = \beta_0+\beta_1 x_1+\beta_2 x_2+\cdots+\beta_k x_k+\epsilon$};
  \node[subtxt] at (Y.south) {$n\times 1$};
  \node[subtxt, align=center] at (X.south) {$n\times p$\\ $(p=k+1)$};
  \node[subtxt] at (beta.south) {$p\times 1$};
  \node[subtxt] at (E.south) {$n\times 1$};
\end{tikzpicture}

\section{向量画图}

\begin{tikzpicture}[
  yshift=-1.3, xshift=0.7, >=stealth,
  axis/.style={->},
  vec/.style={thick, ->},
  every node/.style={color=black}]

  \coordinate (O) at (0,0,0);
  \draw[dotted, draw=darkgreen, fill=darkgreen!20]
    (-2,0,-3.5) -- (3,0,-3.5) -- (4,0,3) -- (-1,0,3) -- cycle;
  \draw[axis] (O) -- +(3, 0, 0);
  \draw[axis] (O) -- +(0, 2, 0);
  \draw[axis] (O) -- +(0, 0, 2);
  %% Vetor y.
  \draw[vec, color=red] (O) -- (3.4, 4, 5.1)
    node [above right] {$y = (3.4, 4, 5.1)^\top$};
  %% Colunas de X.
  \draw[vec, darkgreen] (O) -- (1, 1, 1)
    node [right] {$X_1 = (1, 1, 1)^\top$};
  \draw[vec, darkgreen] (O) -- (0, 1, 2)
    node [left] {$X_2 = (0,1,2)^\top$};
  \end{tikzpicture}



  


 




\section{简单的3d图}

\begin{tikzpicture}
  \begin{axis}[
    title=Example using the mesh parameter,
    hide axis,
    colormap/cool]
    \addplot3[mesh, samples=50, domain=-8:8]
      {sin(deg(sqrt(x^2+y^2)))/sqrt(x^2+y^2)};
    \addlegendentry{$\frac{sin(r)}{r}$}
  \end{axis}
\end{tikzpicture}


\section{简单的函数注释1}

\def\bzero{0.1}
\def\bum{1}
\def\boti{4}
\def\dd{-3.5}

\begin{tikzpicture}[domain=1:8, xscale=1, yscale=5, >=latex]
  \draw[->, line width=1pt] (0,0) -- (10,0) node[below] {$n$};
  \draw[->, line width=1pt] (0,0) -- (0,1.2) node[left] {$CV(n)$};
  \draw[color=green!30!black, thick, smooth, samples=50]
    plot[id=x, domain=0.9:9.7]
    function{\bzero+(\bum-\bzero)/(1+((1-0.15)/0.15)*((x-1)/\boti))};
  \draw[dashed] (1,0) node[below] {$1$} |- (0,\bum)
    node[left] {$b_1 = CV(1)$};
  \draw[dashed] (0,\bzero) node[left]
    {$b_0 = \displaystyle\lim_{n\to\infty} CV(n)$} -- (10,\bzero);
  \draw[dashed] (\boti,0) node[below] {$b_2$} |- (0,0.28)
    node[left] {$b_0+q(b_1-b_0)$};
  \draw [decorate, decoration={brace,amplitude=4pt}]
    (\dd,0.1) -- (\dd,0.28) node [black,midway,left=4pt] {$q$};
  \draw [decorate, decoration={brace,amplitude=4pt}]
    (\dd,0.28) -- (\dd,1) node [black,midway,left=4pt] {$1-q$};
  \node[anchor=base, fill=white] (vg) at (6,1)
    {$\displaystyle CV(n) = b_0+\frac{b_1-b_0}{
        1+\displaystyle\frac{1-q}{q}\cdot\frac{n-1}{b_2-1}}$};
  \path[->] (\boti,0.28) edge[bend left=10]
    node [at end,right,fill=white, text width=5cm, text ragged]
    {tamanho {\'o}timo de parcela correspondente {\`a} $q$.} ++(1,0.2);
\end{tikzpicture}

\section{简单的函数注释2}

\def\ptsize{0.75pt}%
\tikzset{
  mypoints/.style={fill=white,draw=black},
  pil/.style={->, shorten <=2pt, shorten >=2pt}
}%
\begin{tikzpicture}[>=latex, xscale=5, yscale=5]
  \def\thetah{0.3}; \def\thetac{5.6}; %\def\thetac{-2.8}
  \draw[->, thick] (-0.05,0) -- +(1.2,0) node[below] {tempo ($t$)};
  \draw[->, thick] (0,-0.05) -- +(0,1.2) node[left] {tanino ($y$)};
  \def\bet{0.2}
  \def\alp{0.7}
  \def\v{0.14}
  \draw[-, color=green!60!black, ultra thick, samples=100]
    plot[id=x, domain=0:1]  
    function{\bet+\alp*2**(-x/\v)}
    node[color=black, right] {$f(t) = tnp+tp \cdot 2^{-t/\theta_{tmv}}$};
  \draw[|<->|] (-0.1,\bet) -- (-0.1,0.9)
    node[midway, left, text width=2cm, text ragged left]
    {tanino\\ polimerizado\\ ($\theta_{tp}$)};
  \draw[|<->|] (-0.1,0) -- (-0.1,\bet)
    node[midway, left, text width=3.2cm, text ragged left]
    {tanino n\~{a}o\\ polimerizado ($\theta_{tnp}$)};
  \draw[<-, dashed] (0,\bet) -- +(0.18,0) node[right, text width=2cm] {ass\'intota\\ inferior};
  \draw[<-, dashed] (0,\bet+\alp) -- +(0.5,0)
    node[right, text width=2cm] {tanino total\\ ($\theta_{tnp}+\theta_{tp}$)};
  \draw[->,dashed] (\v,\bet+0.5*\alp) -- (\v,0)
    node[below, text width=2cm, text centered]
    {tempo de meia vida ($\theta_{tmv}$)};
  \draw[dashed] (0,\bet+0.5*\alp) -- +(\v,0);
  \coordinate (a1) at (0.3,0.6);
  \path[<-, dashed] (0,\bet+0.5*\alp) edge[bend left=50] (a1);
  \node[below right, text width=3.25cm] at (a1)
    {metade do tanino polimerizado};
  \end{tikzpicture}%


  \section{图像并列比较注释}

  \pgfplotsset{
  myplot/.style={
    width=7cm, height=5cm,
    width=7cm, height=5cm,
    xlabel=$x$,
    ylabel=$y$,
    xtick=\empty,
    xticklabels=\empty,
    yticklabels=\empty,
    samples=50, domain=0:5, smooth, no marks,
    font=\footnotesize,
  }
}

\begin{tikzpicture}[
  declare function={
    reg(\x,\a,\b)=\a+\b*\x;
  },
  declare function={
    gamma(\z)=
    (2.506628274631*sqrt(1/\z)+0.20888568*(1/\z)^(1.5)+
    0.00870357*(1/\z)^(2.5)-(174.2106599*(1/\z)^(3.5))/25920-
    (715.6423511*(1/\z)^(4.5))/1244160)*exp((-ln(1/\z)-1)*\z);
  },
  declare function={
    beta(\a,\b)=gamma(\a)*gamma(\b)/gamma(\a+\b);
  },
  declare function={
    betapdf(\x,\a,\b)=\x^(\a-1)*(1-\x)^(\b-1)/beta(\a,\b);
  },
  >=stealth,
  cx/.style={fill=white, font=\footnotesize},
  pth/.style={draw, ->, color=darkgreen}]

  \begin{axis}[
    myplot,
    extra x ticks={2,5,8},
    extra x tick labels={$x_1$,$x_2$,$x_3$},
    extra tick style={grid=major}]

    \addplot[color=black, thick, domain=0:10, samples=2]
      (x,{reg(x,0,1)});
    \node[cx, anchor=north west] (dp)
      at (axis description cs:0.05,0.95)
      {$[y|x]$: assim\'etrica a esquerda};
    \pgfplotsinvokeforeach{2,5,8}{
      \addplot[domain=0:1, samples=30, fill=darkgreen, opacity=0.5]
      ({#1+0.5*betapdf(x,2,6)},7*x-1.5+#1);
    }
  \end{axis}

  \begin{axis}[
    myplot,
    xshift=6.5cm,
    width=5cm, height=5cm,
    xlabel=Quantis te\'{o}ricos,
    ylabel=Quantis observados]

    \addplot[color=black, dotted, domain=-1:1, samples=2]
      (x,{reg(x,0,1)});
    \addplot[color=black, very thick, domain=-1:1,
      samples=20, color=darkgreen] {0.5-0.5*(x-0.95)^2};
    \node[cx, anchor=north] (dp)
      at (axis description cs:0.5,0.95) {Assimetria a esquerda};
  \end{axis}

  \begin{axis}[
    myplot,
    yshift=4.5cm,
    extra x ticks={2,5,8},
    extra x tick labels={$x_1$,$x_2$,$x_3$},
    extra tick style={grid=major}]

    \addplot[color=black, thick, domain=0:10, samples=2]
      (x,{reg(x,0,1)});
    \node[cx, anchor=south east] (dp)
      at (axis description cs:0.95,0.05)
      {$[y|x]$: assim\'{e}trica a direita};
    \pgfplotsinvokeforeach{2,5,8}{
      \addplot[domain=0:1, samples=30, fill=darkgreen, opacity=0.5]
      ({#1+0.5*betapdf(x,6,2)},7*x-5.5+#1);
    }
  \end{axis}

  \begin{axis}[
    myplot,
    xshift=6.5cm, yshift=4.5cm,
    width=5cm, height=5cm,
    xlabel=Quantis te\'{o}ricos,
    ylabel=Quantis observados]

    \addplot[color=black, dotted, domain=-1:1, samples=2]
      (x,{reg(x,0,1)});
    \addplot[color=black, very thick, domain=-1:1,
      samples=20, color=darkgreen] {-0.5+0.5*(x+0.95)^2};
    \node[cx, anchor=south] (dp)
      at (axis description cs:0.5,0.05) {Assimetria a direita};
  \end{axis}
\end{tikzpicture}
  

  \section{多个分布图像的垂直比较}

  \def\hscale{-1}
\def\stderr{0.35}
\def\fromto{1.8}

\newcommand*{\ListXYvalues}{1/2, 2/0, 3/1.5, 4/0.5}
\newcommand*\pgfplotsinvokeforeachmacro[1]
{\expandafter\pgfplotsinvokeforeach\expandafter{#1}}

\tikzset{ 
  declare function={
    normal(\m,\s)=1/(2*\s*sqrt(pi))*exp(-(x-\m)^2/(2*\s^2));
  },
  >=stealth,
  cx/.style={fill=white, font=\footnotesize},
  pth/.style={draw, ->, color=darkgreen},
  halves/.style={samples=30, fill opacity=0.5, draw=none}
}

\pgfplotsset{
  my plot/.code args={#1/#2}{%
    \addplot[domain=-\fromto:\fromto]
    ({#1+\hscale*normal(0,\stderr)}, x+#2);
    \addplot[halves, domain=-\fromto:0, fill=darkgreen!50]
    ({#1+\hscale*normal(0,\stderr)}, x+#2) -- (axis cs: #1, #2);
    \addplot[halves, domain=0:\fromto, fill=darkgreen]
    ({#1+\hscale*normal(0,\stderr)}, x+#2) -- (axis cs: #1, #2);
    \node[rotate=90, below] at (axis cs: #1, #2) {$\mu+\tau_{#1}$};
  }
}

\begin{tikzpicture}[rotate=-90]
  \begin{axis}[
    width=8cm, height=9cm,
    xlabel=Tratamentos ($i$),
    ylabel=Vari\'{a}vel resposta ($Y$),
    x label style={rotate=180},
    xtick=\empty, xticklabels=\empty, yticklabels=\empty,
    extra x ticks={1,2,3,4},
    extra x tick labels={$i=1$,$i=2$,$i=3$,$i=4$},
    x tick label style={rotate=90, anchor=east},
    extra tick style={grid=major},
    samples=20, domain=-0:4, smooth]

    \node[cx, rotate=90, anchor=north west] (eq)
    at (axis description cs: 0.02, 0.01)
    {$[Y|i]\sim$ Normal($\mu_i=\mu+\tau_i$, $\sigma^2$)};

    \pgfplotsinvokeforeachmacro\ListXYvalues{
      \pgfplotsset{my plot={#1}}
    }

  \end{axis}
\end{tikzpicture}

\section{简单分布函数,加色}

\makeatletter
\pgfdeclarepatternformonly[%
\hatchdistance,\hatchthickness]{flexible hatch}
{\pgfqpoint{0pt}{0pt}}
{\pgfqpoint{\hatchdistance}{\hatchdistance}}
{\pgfpoint{\hatchdistance-1pt}{\hatchdistance-1pt}}%
{
  \pgfsetcolor{\tikz@pattern@color}
  \pgfsetlinewidth{\hatchthickness}
  \pgfpathmoveto{\pgfqpoint{0pt}{0pt}}
  \pgfpathlineto{\pgfqpoint{\hatchdistance}{\hatchdistance}}
  \pgfusepath{stroke}
}
\makeatother

\begin{tikzpicture}[
  hatch distance/.store in=\hatchdistance,
  hatch distance=10pt,
  hatch thickness/.store in=\hatchthickness,
  hatch thickness=2pt]

  \begin{axis}[
    width=10cm, height=6cm,
    xlabel={z},
    axis on top,
    legend style={
      draw=none, legend cell align=left, legend plot pos=left}]

    \addplot [mark=none, domain=0:1, samples=100,
    pattern=flexible hatch, hatch distance=10pt,
    hatch thickness=2pt, draw=darkgreen, pattern color=darkgreen,
    area legend] {1/sqrt(2*pi)*exp(-x^2/2)} \closedcycle;
    \addlegendentry{Intervalo 1}

    \addplot [mark=none, domain=-2:-0.5, samples=100,
    pattern=flexible hatch, hatch distance=5pt,
    hatch thickness=0.75pt, draw=darkgreen, pattern color=darkgreen,
    area legend] {1/sqrt(2*pi)*exp(-x^2/2)} \closedcycle;    
    \addlegendentry{Intervalo 2}

    \addplot[color=black, thick, domain=-5:5, samples=100]
    {1/sqrt(2*pi)*exp(-x^2/2)};
    \addlegendentry{z}

  \end{axis}
\end{tikzpicture}

\section{概率决策树}

\tikzset{
  level 1/.style={level distance=3.5cm, sibling distance=4.5cm},
  level 2/.style={level distance=3.5cm, sibling distance=2.2cm},
  bag/.style={text width=8em, text centered, anchor=west,
    fill=gray!50, rounded corners, minimum height=3em},
  end/.style={circle, minimum width=3pt, fill, inner sep=0pt,
    anchor=west}
}

\begin{tikzpicture}[grow=right, sloped, ->, >=stealth']
  \node[bag] {falha na \\ superf\'icie?}
  child {
    node[bag] {falha no \\ funcionamento?}        
    child {
      node[end, label=right:
      {$\Pr(S^c\cap F^c)=\dfrac{360}{400}\cdot
        \dfrac{342}{360}=\dfrac{342}{400}$}] {}
      edge from parent
      node[above] {n\~ao $(F^c)$}
      node[below] {$\Pr(F^c|S^c)=\frac{342}{360}$}
    }
    child {
      node[end, label=right:
      {$\Pr(S^c\cap F)=\dfrac{360}{400}\cdot
        \dfrac{18}{360}=\dfrac{18}{400}$}] {}
      edge from parent
      node[above] {sim ($F$)}
      node[below] {$\Pr(F|S^c)=\frac{18}{360}$}
    }
    edge from parent 
    node[above] {n\~ao ($S^c$)}
    node[below] {$\Pr(S^c)=\frac{360}{400}$}
  }
  child {
    node[bag] {falha no \\ funcionamento?}        
    child {
      node[end, label=right:
      {$\Pr(S\cap F^c)=\dfrac{40}{400}\cdot
        \dfrac{30}{40}=\dfrac{30}{400}$}] {}
      edge from parent
      node[above] {n\~ao ($F^c$)}
      node[below] {$\Pr(F^c|S)=\frac{30}{40}$}
    }
    child {
      node[end, label=right:
      {$\Pr(S\cap F)=\dfrac{40}{400}\cdot
        \dfrac{10}{40}=\dfrac{10}{400}$}] {}
      edge from parent
      node[above] {sim ($F$)}
      node[below] {$\Pr(F|S)=\frac{10}{40}$}
    }
    edge from parent         
    node[above] {sim ($S$)}
    node[below] {$\Pr(S)=\frac{40}{400}$}
  };
\end{tikzpicture}

\section{多个简单函数的关联}


\tikzset{
  state/.style={
    rectangle, rounded corners, draw=black, very thick,
    minimum height=2em, inner sep=10pt, text centered,
  },
}

\begin{tikzpicture}[->, >=latex, line width=1pt]
  \node[state] (f1) {
    $f_1(t; A, V, C, D) = \displaystyle \frac{A}{1+(V/t)^C}+D\cdot t$
  };
  \node[state, yshift=-3cm, left of=f1,
  node distance=5.5cm, anchor=center] (f2) {
    $f_2(t; A, V, C) = \displaystyle \frac{A}{1+(V/t)^C}$
  };
  \node[state, yshift=-3cm, right of=f1,
  node distance=5.5cm, anchor=center] (f3) {
    $f_3(t; A, V, D) = \displaystyle \frac{A}{1+(V/t)}+D\cdot t$
  };
  \node[state, yshift=-3cm, right of=f2,
  node distance=5.5cm, anchor=center] (f4) {
    $f_4(t; A, V) = \displaystyle \frac{A}{1+(V/t)}$
  };
  \node[state, yshift=-3cm, right of=f3,
  node distance=2.5cm, anchor=center] (f5) {
    $f_5(t; A, D) = A+D\cdot t$\\
  };
  \path
  (f1) edge[bend right=20] node[midway, left] {$D = 0$} (f2)
  (f1) edge[bend left=20] node[midway, right] {$C = 1$} (f3)
  (f2) edge[bend right=30] node[midway, left] {$C = 1$} (f4)
  (f3) edge[bend left=30] node[midway, right] {$D = 0$} (f4)
  (f3) edge[bend right=10] node[midway, right] {$V = 0$} (f5)
  (f1) edge[bend left=20] node[midway, left]
    {$D = 0 \textrm{ e } C = 1$} (f4);
  \begin{scope}[xshift=-1.5cm, yshift=1cm]
    \draw[<->] (0,2) |- (3.25,0);
    \draw[-, color=green!30!black, thick]
      plot[id=x, domain=-0.01:3]
      function{1/(1+(0.5/x)**4)+0.25*x};
  \end{scope}
  \begin{scope}[xshift=-9cm, yshift=-2cm]
    \draw[<->] (0,2) |- (3.25,0);
    \draw[-, color=green!30!black, thick]
      plot[id=x, domain=-0.01:3]
      function{1.3/(1+(0.5/x)**4)};
  \end{scope}
  \begin{scope}[xshift=6cm, yshift=-2cm]
    \draw[<->] (0,2) |- (3.25,0);
    \draw[-, color=green!30!black, thick]
      plot[id=x, domain=-0.01:3]
      function{1/(1+(0.5/x))+0.25*x};
  \end{scope}
  \begin{scope}[xshift=-1.5cm, yshift=-9cm]
    \draw[<->] (0,2) |- (3.25,0);
    \draw[-, color=green!30!black, thick]
      plot[id=x, domain=-0.01:3]
      function{1.6/(1+(0.3/x))};
  \end{scope}
  \begin{scope}[xshift=6.5cm, yshift=-9cm]
    \draw[<->] (0,2) |- (3.25,0);
    \draw[-, color=green!30!black, thick]
      plot[id=x, domain=-0.01:3]
      function{0.7+0.3*x};
  \end{scope}
\end{tikzpicture}

\section{水平排列物体加标注}


\def\A{\clubsuit}
\def\B{\heartsuit}

\begin{tikzpicture}
  \draw (-2,0) rectangle (7,3) node[above right] at (-2,0) {$\Omega$};
  \node[above] at (2.5,3) {elementos de $\Omega$};
  \node (111) at (6,2) {};
  \node (110) at (5,1) {};
  \node (011) at (4,2) {};
  \node (101) at (3,1) {};
  \node (001) at (2,2) {};
  \node (100) at (1,1) {};
  \node (010) at (0,2) {};
  \node (000) at (-1,1) {};
  \node[left, text width=3em, text centered] (arrowstart) at (-2,-2)
    {valores de $X$ (U\$)};
  \node (arrowend) at (7,-2) {$\mathbb{R}$};
  \draw (111) circle (2pt) node[above] {$\A\A\A$};
  \draw (110) circle (2pt) node[above] {$\A\A\B$};
  \draw (011) circle (2pt) node[above] {$\B\A\A$};
  \draw (101) circle (2pt) node[above] {$\A\B\A$};
  \draw (001) circle (2pt) node[above] {$\B\B\A$};
  \draw (100) circle (2pt) node[above] {$\A\B\B$};
  \draw (010) circle (2pt) node[above] {$\B\A\B$};
  \draw (000) circle (2pt) node[above] {$\B\B\B$};
  \draw[->] (arrowstart) -- (arrowend);
  \foreach \x/\y in {0/0,1/50,2/100,3/250,4/500,5/1000}{
    \draw (\x,-1.8) -- (\x,-2) node[anchor=north] {\y};
  }
  \path [->,>=stealth']
    (111) edge[bend left] (5,-2)
    (110) edge[bend left] (4,-2)
    (011) edge[bend left] (4,-2)
    (101) edge[bend left] (3,-2)
    (001) edge[bend left] (2,-2)
    (100) edge[bend right] (2,-2)
    (010) edge[bend right] (1,-2)
    (000) edge[bend right] (0,-2);
\end{tikzpicture}


\section{水平排列物体加色加标注}

\tikzset{
  >=stealth',
  every path/.style={->, color=darkgreen, thin},
  every node/.style={color=black},
  cir/.style={draw, rounded corners=2pt,
    color=darkgreen, inner sep=7pt},
  mtx/.style={matrix of math nodes, column sep=1ex},
  hltr/.style={opacity=0.2,
    rounded corners=2pt, inner sep=-1pt, fill=darkgreen,
    minimum width=4ex},
  ar/.style={out=45, in=-45},
  node distance=0.5ex
}

\begin{tikzpicture}[
  background rectangle/.style={fill=blue!10!white},
  show background rectangle]

  \matrix[mtx] (A) at (-5,2){
    4 & 7 & 8 & 8 & 9 & 10 & 10 & 11 & 11 & 12 & 12 & 13 & 17 \\
  };
  %% Mediana.
  \node[hltr, fit=(A-1-7)] (q2) {};
  \node[above=of q2] (q2text) {$Q2$};
  %% 1 quartil.
  \node[hltr, fit=(A-1-4)] (q1) {};
  \node[above=of q1] (q1text) {$Q1$};
  %% 3 quartil.
  \node[hltr, fit=(A-1-10)] (q3) {};
  \node[above=of q3] (q3text) {$Q3$};
  \draw[-] (q2) -- ++(1ex,-2.5ex) -|
    node[near start, below] {metade direita} (A-1-13);
  \draw[-] (q2) -- ++(-1ex,-2.5ex) -|
    node[near start, below] {metade esquerda} (A-1-1);
  \node[above=of q2text] {Amostra de tamanho \'impar/\'impar};

  \matrix (B) at (5,2) [mtx] {
    7 & 8 & 8 & 9 & 10 & 10 & 11 & 11 & 12 & 12 & 13 \\
  };
  %% Mediana.
  \node[hltr, fit=(B-1-6)] (q2) {};
  \node[above=of q2] (q2text) {$Q2$};
  %% 1 quartil.
  \node[hltr, fit=(B-1-3)(B-1-4)] (q1) {};
  \node[above=of q1] (q1text) {$Q1$};
  %% 3 quartil.
  \node[hltr, fit=(B-1-8)(B-1-9)] (q3) {};
  \node[above=of q3] (q3text) {$Q3$};
  \draw[-] (q2) -- ++(1ex,-2.5ex) -|
    node[near start, below] {metade direita} (B-1-11);
  \draw[-] (q2) -- ++(-1ex,-2.5ex) -|
    node[near start, below] {metade esquerda} (B-1-1);
  \node[above=of q2text] {Amostra de tamanho \'impar/par};

  \matrix (D) at (-5,-2) [mtx] {
    2 & 4 & 7 & 8 & 8 & 9 & 10 & 10 & 11 & 11 & 12 & 12 & 13 & 17 \\
  };
  %% Mediana.
  \node[hltr, fit=(D-1-7)(D-1-8)] (q2) {};
  \node[above=of q2] (q2text) {$Q2$};
  %% 1 quartil.
  \node[hltr, fit=(D-1-4)] (q1) {};
  \node[above=of q1] (q1text) {$Q1$};
  %% 3 quartil.
  \node[hltr, fit=(D-1-11)] (q3) {};
  \node[above=of q3] (q3text) {$Q3$};
  \draw[-] (q2) -- ++(1ex,-2.5ex) -|
    node[near start, below] {metade direita} (D-1-14);
  \draw[-] (q2) -- ++(-1ex,-2.5ex) -|
    node[near start, below] {metade esquerda} (D-1-1);
  \node[above=of q2text] {Amostra de tamanho par/\'impar};

  \matrix (C) at (5,-2) [mtx] {
    4 & 7 & 8 & 8 & 9 & 10 & 10 & 11 & 11 & 12 & 12 & 13 \\
  };
  %% Mediana.
  \node[hltr, fit=(C-1-6)(C-1-7)] (q2) {};
  \node[above=of q2] (q2text) {$Q2$};
  %% 1 quartil.
  \node[hltr, fit=(C-1-3)(C-1-4)] (q1) {};
  \node[above=of q1] (q1text) {$Q1$};
  %% 3 quartil.
  \node[hltr, fit=(C-1-9)(C-1-10)] (q3) {};
  \node[above=of q3] (q3text) {$Q3$};
  \draw[-] (q2) -- ++(1ex,-2.5ex) -|
    node[near start, below] {metade direita} (C-1-12);
  \draw[-] (q2) -- ++(-1ex,-2.5ex) -|
    node[near start, below] {metade esquerda} (C-1-1);
  \node[above=of q2text] {Amostra de tamanho par/par};

  % \node[fit=(A)(B)(C)(D), draw] {};
  % \draw[help lines,step=1] (-10,-5) grid (10,5);
\end{tikzpicture}


\section{回归函数图像1}


\begin{tikzpicture}[
  >=stealth,
  cx/.style={color=black, text width=2cm, font=\footnotesize},
  pil/.style={->, color=darkgreen, rounded corners},
  every node/.style={color=black}]

  \begin{axis}[
    width=9cm, height=6cm,
    xlabel=$x$: preditora, ylabel=$y$: resposta]

    \addplot[color=darkgreen, thick, mark=none, domain=0:13] {2+3.4*x}
      node[pos=0.8, sloped, below] {$f(x)=\beta_0+\beta_1 x$};
    \addplot[color=darkgreen, only marks, mark=o]
      coordinates {
        (1,8)
        (2,7)
        (3,11)
        (4,20)
        (5,12)
        (6,25)
        (7,24)
      };
    \coordinate (pontoreta) at (axis cs: 10, 36);
    \coordinate (yfit) at (axis cs: 5, 19);
    \coordinate (yobs) at (axis cs: 5, 12);
    \node[cx, align=flush right] (compdet)
      at (axis description cs: 0.35, 0.9)
      {Componente determin\'{i}stico};
    \draw[pil] (compdet) -| (pontoreta);
    \draw[pil, <->] (yfit) -- node[cx, right]
      {Componente aleat\'{o}rio $(\epsilon)$} (yobs);

  \end{axis}
\end{tikzpicture}


\section{回归函数图像2}

\begin{tikzpicture}[
  >=stealth,
  tx/.style={circle=1pt, inner sep=0pt, fill=red}]

  \begin{axis}[
    width=7cm, height=7cm,
    xmin=-3, xmax=3,
    ymin=-3, ymax=3,
    extra x ticks={0,-1.8,1.8},
    extra y ticks={0,-1.8,1.8},
    extra x tick labels={$\hat{\beta}_0$},
    extra y tick labels={$\hat{\beta}_1$},
    extra tick style={grid=major},
    xticklabels=\empty,
    yticklabels=\empty]

    \draw[fill=none, opacity=1, dashed, color=red]
      (axis cs:-1.8,-1.8) rectangle (axis cs:1.8,1.8);
    \draw[fill=none, opacity=1, dashed, color=blue]
      (axis cs:-2,-2) rectangle (axis cs:2,2);
    \path[draw, ->] (axis cs:1.8,1.6)
      to[out=45, in=180] (axis cs:2.3,1.6)
      node[right] {$t$};
    \path[draw, ->] (axis cs:2,1)
      to[out=45, in=90] (axis cs:2.5,0.8)
      node[below] {$(pF)^{\frac{1}{2}}$};

    \addplot[only marks, mark=o] coordinates { (0,0) };

    \draw[darkgreen, fill, opacity=0.5]

    \pgfextra{
      \pgfpathellipse{\pgfplotspointaxisxy{0}{0}}
      {\pgfplotspointaxisdirectionxy{-2}{1.75}}
      {\pgfplotspointaxisdirectionxy{0}{1}}
    };

    \node[anchor=north east, fill=white] (eq)
      at (axis description cs:0.98,0.98)
      {$RC_{1-\alpha}(\beta_0,\beta_1)$};

    \coordinate (in) at (axis description cs:0.55,0.55);
    \path[draw, ->] (in) to[out=45, in=-90] (eq);
    \draw[|<->|] (axis cs: -1.8,-2.2) -- node[below]
      {$IC_{1-\alpha}(\beta_0)$} (axis cs:1.8,-2.2); 
    \draw[|<->|] (axis cs: -2.2,-1.8) -- node[rotate=90, above]
      {$IC_{1-\alpha}(\beta_1)$} (axis cs:-2.2,1.8); 

    \node[tx] at (axis cs: 1.4,-1.4) {\tiny A};
    \node[tx] at (axis cs: 1.4,0.4) {\tiny B};
    \node[tx] at (axis cs: 1.9,-1.5) {\tiny C};
    \node[tx] at (axis cs: 1.9,-1.9) {\tiny D};

  \end{axis}
\end{tikzpicture}



\section{回归函数图像3}


\begin{filecontents*}{galton.dat}
     x      y
1.9042 1.9618
1.6095 1.4159
1.6750 1.3323
1.5793 1.5033
1.7846 1.6441
1.6580 1.7589
1.8022 2.0789
1.8168 1.5724
1.5277 1.7982
2.2290 2.3234
1.8467 1.8012
1.2815 1.3210
1.5189 1.5609
1.3392 1.2647
1.4282 1.4376
2.3150 2.1893
1.0115 0.9544
2.1810 2.2179
0.9433 0.5742
2.2594 1.9946
2.2742 2.4213
1.5422 1.8452
1.4814 1.7540
2.0522 2.0247
2.0164 1.8147
1.3238 1.6789
2.3727 2.0102
1.6782 1.8770
1.7112 1.7508
1.4119 1.4921
1.3997 1.3480
1.8825 1.8111
1.4035 1.7164
1.9845 1.9546
1.2613 1.4185
2.0998 2.3210
1.6218 1.8376
1.9008 1.9458
1.1555 1.2630
1.3793 1.3378
1.6014 1.7404
1.4219 1.8182
1.5087 1.3820
1.7182 1.6391
1.8483 1.7016
1.7815 1.6210
2.0682 2.0595
1.6333 1.9390
1.8845 1.8987
2.1268 2.2169
\end{filecontents*}

\begin{tikzpicture}[>=stealth, font=\small]

  \begin{axis}[
    unit vector ratio*=1 1 1, % Para ter iso nos eixos.
    width=9cm,
    grid=major,
    grid style={dashed, gray!30},
    xlabel=M\'edia da altura dos pais,
    ylabel=M\'edia da altura dos filhos,
    xtick={1,1.25,...,3},
    ytick={1,1.25,...,2.5},
    xticklabels=\empty,
    yticklabels=\empty,
    xmin=1, xmax=3,
    ymin=1.1, ymax=2.6,
    domain=1.1:2.5,
    legend pos=south east,
    legend style={draw=none, legend cell align=left}]

    \addplot[only marks, mark=o] table {galton.dat};
    \addlegendentry{observa\c{c}\~oes};
    \addplot[thick] {x};
    \addlegendentry{reta 1:1};
    \addplot[thick, color=darkgreen] {0.2+0.89*x};
    \addlegendentry{ajuste};
    \addplot[domain=1:3, mark=none, samples=2, dashed] {1.72};

    \path[->, , shorten <=2pt, shorten >=2pt]
      (axis cs: 2.3, 2.3) edge[bend left]
      node[right, text width=1.8cm, font=\footnotesize, pos=0.65]
      {Regress\~ao para a m\'edia} (axis cs: 2.4, 1.72);

  \end{axis}
\end{tikzpicture}

\section{回归图解}

\def\hscale{1}
\def\stderr{0.5}
\def\fromto{2}

\begin{tikzpicture}[
  >=stealth,
  cx/.style={fill=white, font=\footnotesize},
  pth/.style={draw, ->, color=darkgreen},
  halves/.style={samples=30, fill opacity=0.5, draw=none},
  % declare function={
  %   normal(\m,\s)=1/(2*\s*sqrt(pi))*exp(-(x-\m)^2/(2*\s^2));
  % },
  declare function={
    reg(\x,\a,\b)=\a+\b*\x;
  }]

  \begin{axis}[
    width=10cm, height=7cm,
    xlabel=$x$: preditora,
    ylabel=$y$: resposta,
    xtick=\empty,
    xticklabels=\empty,
    yticklabels=\empty,
    extra x ticks={1,2,3},
    extra x tick labels={$x_1$,$x_2$,$x_3$},
    extra tick style={grid=major},
    samples=30, domain=-0:4,
    ymin=-2, ymax=6, xmin=-1, xmax=5]

    \addplot[color=black, thick, samples=2] (x, {reg(x,0,1)});

    \node[cx] (dp) at (axis description cs: 0.65, 0.1)
      {\footnotesize $[y|x]$};
    \node[cx, anchor=south west] (eq)
      at (axis description cs: 0.02, 0.03) {$E(y) = \beta_0+\beta_1 x$};
    \path[pth, shorten >=2pt] (eq) to[out=90, in=-135] (axis cs: 0, 0);

    \pgfplotsinvokeforeach{1,2,3}{
      \addplot[domain=-\fromto:\fromto]
        ({#1+\hscale*normal(0,\stderr)}, x+#1);
      \addplot[halves, domain=-\fromto:0, fill=darkgreen!50]
        ({#1+\hscale*normal(0,\stderr)}, x+#1) -- (axis cs:#1,#1);
      \addplot[halves, domain=0:\fromto, fill=darkgreen]
        ({#1+\hscale*normal(0,\stderr)}, x+#1) -- (axis cs: #1, #1);
      \draw[dashed] (axis cs:#1,#1) -- (axis cs: #1+0.6, #1);
      \path[pth] (dp) to[out=130, in=-45] (axis cs: #1+0.4, #1-\stderr);
      \node[cx] (x#1) at (axis cs: #1-0.7, #1+2) {$E(y|x=x_{#1})$};
      \path[pth] (x#1) to[out=-60, in=180] (axis cs: #1, #1);
    }
  \end{axis}
\end{tikzpicture}



\section{日历}

\makeatletter
% Define our own style
\tikzstyle{week list sunday}=[
  % Note that we cannot extend from week list,
  % the execute before day scope is cumulative
  execute before day scope={%
    \ifdate{day of month=1}{\ifdate{equals=\pgfcalendarbeginiso}{}{
        % On first of month, except when first date in calendar.
        \pgfmathsetlength{\pgf@y}{\tikz@lib@cal@month@yshift}%
        \pgftransformyshift{-\pgf@y}
      }}{}%
  },
  execute at begin day scope={%
    % Because for TikZ Monday is 0 and Sunday is 6,
    % we can't directly use \pgfcalendercurrentweekday,
    % but instead we define \c@pgf@counta (basically) as:
    % (\pgfcalendercurrentweekday + 1) % 7
    \pgfmathsetlength\pgf@x{\tikz@lib@cal@xshift}%
    \ifnum\pgfcalendarcurrentweekday=6
    \c@pgf@counta=0
    \else
    \c@pgf@counta=\pgfcalendarcurrentweekday
    \advance\c@pgf@counta by 1
    \fi
    \pgf@x=\c@pgf@counta\pgf@x
    % Shift to the right position for the day.
    \pgftransformxshift{\pgf@x}
  },
  execute after day scope={
    % Week is done, shift to the next line.
    \ifdate{Saturday}{
      \pgfmathsetlength{\pgf@y}{\tikz@lib@cal@yshift}%
      \pgftransformyshift{-\pgf@y}
    }{}%
  },
  % This should be defined, glancing from the source code.
  tikz@lib@cal@width=7
]
\tikzoption{day headings}{\tikzstyle{day heading}=[#1]}
\tikzstyle{day heading}=[]
\tikzstyle{day letter headings}=[
  execute before day scope={\ifdate{day of month=1}{%
      \pgfmathsetlength{\pgf@ya}{\tikz@lib@cal@yshift}%
      \pgfmathsetlength\pgf@xa{\tikz@lib@cal@xshift}%
      \pgftransformyshift{-\pgf@ya}
      \foreach \d/\l in {0/D,1/S,2/T,3/Q,4/Q,5/S,6/S} {
        \pgf@xa=\d\pgf@xa%
        \pgftransformxshift{\pgf@xa}%
        \pgftransformyshift{\pgf@ya}%
        \node[every day, day heading]{\tiny\l};%
      } 
    }{}%
  }%
]
\makeatother

\tikzstyle{labest}=[font=\footnotesize, fill=orange!50, inner sep=2pt]
\tikzstyle{evento}=[font=\footnotesize, fill=red!50, inner sep=2pt]

% The actual calendar is now rather easy:
\begin{tikzpicture}[
  every calendar/.style={
    month label above centered,
    % day letter headings,
    month text={\textit{\%mt}},
    if={(Sunday) [blue!75]},
    if={(Saturday) [black!50]},
    week list sunday,
    day yshift=1em, day xshift=1.25em}]

  % % Layout 3 colunas e 4 linhas.
  % \matrix[column sep=0.1em, row sep=0.2em] {
  %   \calendar[dates=2016-01-01 to 2016-01-last]; &
  %   \calendar[dates=2016-02-01 to 2016-02-last]
  %     if (equals=02-29) [orange]; &
  %   \calendar[dates=2016-03-01 to 2016-03-last]; \\
  %   \calendar[dates=2016-04-01 to 2016-04-last]; &
  %   \calendar[dates=2016-05-01 to 2016-05-last]
  %     if (between=05-23 and 05-25) [red]
  %     if (equals=05-06) [orange]; &
  %   \calendar[dates=2016-06-01 to 2016-06-last]; \\
  %   \calendar[dates=2016-07-01 to 2016-07-last]
  %     if (between=07-24 and 07-29) [red]; &
  %   \calendar[dates=2016-08-01 to 2016-08-last]; &
  %   \calendar[dates=2016-09-01 to 2016-09-last]; \\
  %   \calendar[dates=2016-10-01 to 2016-10-last]
  %     if (between=10-04 and 10-05) [red]; &
  %   \calendar[dates=2016-11-01 to 2016-11-last]; &
  %   \calendar[dates=2016-12-01 to 2016-12-last]; \\
  % };

  % Layout 4 colunas e 3 linhas.
  \matrix[column sep=0.1em, row sep=0em] {
    \calendar[dates=2016-01-01 to 2016-01-last]; &
    \calendar (Fev) [dates=2016-02-01 to 2016-02-last]
      if (equals=02-29) [orange]; &
    \calendar[dates=2016-03-01 to 2016-03-last]; &
    \calendar[dates=2016-04-01 to 2016-04-last]; \\
    \calendar (Mai) [dates=2016-05-01 to 2016-05-last]
      if (between=05-23 and 05-25) [red]
      if (equals=05-06) [orange]; &
    \calendar[dates=2016-06-01 to 2016-06-last]; &
    \calendar (Jul) [dates=2016-07-01 to 2016-07-last]
      if (between=07-24 and 07-29) [red]; &
    \calendar (Ago) [dates=2016-08-01 to 2016-08-last]
      if (equals=08-01) [orange]; \\
    \calendar[dates=2016-09-01 to 2016-09-last]; &
    \calendar (Out) [dates=2016-10-01 to 2016-10-last]
      if (between=10-04 and 10-05) [red]
      if (equals=10-07) [orange]; &
    \calendar[dates=2016-11-01 to 2016-11-last]; &
    \calendar[dates=2016-12-01 to 2016-12-last]; \\
  };

  \draw[black] (Fev-2016-02-29) |- +(0.3,-0.4)
    node [labest, right] {$\blacktriangleright$ \textit{labest}};
  \draw[black] (Mai-2016-05-06) |- +(0.3,0.4)
    node [labest, right] {{\tiny $\blacksquare$} \textit{labest}};
  \draw[black] (Ago-2016-08-01) -| +(-0.4,0.4)
    node [labest, above] {$\blacktriangleright$ \textit{labest}};
  \draw[black] (Out-2016-10-07) |- +(0.1,0.8)
    node [labest, right] {{\tiny $\blacksquare$} \textit{labest}};

  \draw[black] (Mai-2016-05-25) |- +(0.1,-0.5)
    node [evento, right] {\textit{RBRAS}};
  \draw[black] (Jul-2016-07-28) |- +(0.1,-0.4)
    node [evento, right] {\textit{SINAPE}};
  \draw[black] (Out-2016-10-04) |- +(-0.1,0.4)
    node [evento, left] {\textit{SIEPE}};

\end{tikzpicture}


\section{韦恩图}

\begin{tikzpicture}[
  every path/.style = {
   ->,
   > = stealth, 
   rounded corners},
  state/.style = {
    fill = white,
    text centered
  },
  node distance=1.25cm]

  \definecolor{color1}{HTML}{E7AD00}
  \definecolor{color2}{HTML}{A5CC19}
  \definecolor{color3}{HTML}{33B29A}
  \definecolor{color4}{HTML}{3380FF}
  \definecolor{color5}{HTML}{9033FF}
  \definecolor{color6}{HTML}{E5003D}

\begin{scope}[
  opacity = 1,
  fill opacity = 0.25,
  text opacity = 1,
  text width = 6em,
  text centered]

  \def\firstcircle{(90:2.75cm) circle (3.5cm)}
  \def\secondcircle{(210:2.75cm) circle (3.5cm)}
  \def\thirdcircle{(330:2.75cm) circle (3.5cm)}
  \draw [fill = color6] \firstcircle
    node [above = 10em] {Conhecimento de neg{\' o}cios};
  \draw [fill = color4] \secondcircle
    node [below left = 10em] {Matem{\' a}tica \& Estat{\' i}stica};
  \draw [fill = color2] \thirdcircle
    node [below right = 10em] {Computa{\c c}{\~a}o};

\end{scope}

\begin{scope}

  \node[state] (intuir) at (90:4.5) {Intuir};
  \node[state] (formular) at (60:3.5) {Formular};
  \node[state] (desenhar) at (30:2) {Desenhar};
  \node[state] (coletar)  at (-10:4)  {Coletar};
  \node[state] (armazenar) at (-30:5) {Armazenar};
  \node[state] (importar) at (-55:4.5) {Importar};
  \node[state] (manipular) at (-70:2.75) {Arrumar};
  \node[state] (transformar) at (-90:1.5) {Transformar};
  \node[state] (visualizar) at (-140:4.5) {Visualizar};
  \node[state] (modelar) at (-170:4) {Modelar};
  \node[state] (comunicar) at (150:2) {Compreender};
  \node[state] (agir) at (120:3.5) {Agir};

  \path[draw] (formular) edge[out=-90, in=90] (desenhar);
  \path[draw] (desenhar) edge[out=-90, in=90] (coletar);
  \path[draw] (coletar) edge[out=-70, in=90] (armazenar);
  \path[draw] (armazenar) edge[out=-90, in=0] (importar);
  \path[draw] (importar) edge[out=180, in=-90] (manipular);
  \path[draw] (manipular) edge[out=90, in=-90] (transformar);
  \path[draw] (transformar) edge[out=-120, in=0] (visualizar);
  \path[draw] (visualizar) edge[out=160, in=210] (modelar);
  \path[draw] (modelar) edge[out=0, in=140] (transformar);
  \path[draw] (modelar) edge[out=90, in=-90] (comunicar);
  \path[draw] (comunicar) edge[out=90, in=-90] (agir);
  \path[draw] (agir) edge[out=90, in=180] (intuir);
  \path[draw] (intuir) edge[out=0, in=90] (formular);

\end{scope}

\end{tikzpicture}%






\section{3D 正方形}

\begin{tikzpicture}[%
  scale = 2,
  ->,
  thick,
  z = {(0.45, 0.25)},
  node distance = 2em,
  vertex/.style = {circle, minimum size = 5pt, inner sep = 0pt,
    draw = black, fill = black},
  axial/.style = {rectangle, minimum size = 20pt,
    inner sep = 0pt, fill = gray!30},
  edge/.style = {draw, thick, -, black},
  rotu/.style = {midway},
  sinal/.style = {inner sep = 1pt, thin, opacity = 0.4,
    fill = blue, circle, text opacity = 1},
  pointminus/.style = {draw = blue, fill = blue},
  pointplus/.style = {draw = orange, fill = orange},
  faceminus/.style = {blue, opacity = 0.4},
  faceplus/.style = {orange, opacity = 0.4}
  ]

  \def\dist{0.1}
  \def\cube{
    % Vertices.
    \coordinate (v0) at (0, 0, 0);
    \coordinate (v1) at (0, 1, 0);
    \coordinate (v2) at (1, 0, 0);
    \coordinate (v3) at (1, 1, 0);
    \coordinate (v4) at (0, 0, 1);
    \coordinate (v5) at (0, 1, 1);
    \coordinate (v6) at (1, 0, 1);
    \coordinate (v7) at (1, 1, 1);

    % Edges.
    \draw[edge] (v0) -- (v1) -- (v3) -- (v2) -- (v0);
    \draw[edge] (v0) -- (v4) -- (v5) -- (v1);
    \draw[edge] (v2) -- (v6) -- (v7) -- (v3);
    \draw[edge] (v4) -- (v6);
    \draw[edge] (v5) -- (v7);
  } % \cube

  % A effect.
  \begin{scope}[]
    \cube{};
    \foreach \i in {0,...,7}{ \draw[fill= black] (v\i) circle (1.5pt); }
    \node at (0.25, 1.25, 1) {$2^3 = 8$};

    % Axis text.
    \path (v0) -- node[midway, below=1.5em] {$A$} (v2);
    \path (v0) -- node[midway, left=1.5em] {$B$} (v1);
    \path (v2) -- node[midway, right=3em] {$C$} (v6);

    % Axis text.
    \node[below of=v0, sinal, fill = blue] {$-$};
    \node[below of=v2, sinal, fill = orange] {$+$};
    \node[left of=v0, sinal, fill = blue] {$-$};
    \node[left of=v1, sinal, fill = orange] {$+$};
    \node[right = 3em of v2, sinal, fill = blue] {$-$};
    \node[right = 3em of v6, sinal, fill = orange] {$+$};
  \end{scope}

  % A effect.
  \begin{scope}[xshift = -2.5cm, yshift = -2.5cm]
    \cube{};
    \fill[faceminus]
    (v0.center) -- (v1.center) -- (v5.center) -- (v4.center) -- cycle;
    \foreach \i in {0, 1, 5, 4}{
      \draw[pointminus] (v\i) circle (1.5pt); }
    \fill[faceplus]
    (v2.center) -- (v3.center) -- (v7.center) -- (v6.center) -- cycle;
    \foreach \i in {2, 3, 7, 6}{
      \draw[pointplus] (v\i) circle (1.5pt); }
    \node at (0.25, 1.25, 1) {A};
  \end{scope}

  % B effect.
  \begin{scope}[xshift = 0.0cm, yshift = -2.5cm]
    \cube{};
    \fill[faceminus]
    (v0.center) -- (v4.center) -- (v6.center) -- (v2.center) -- cycle;
    \foreach \i in {0, 4, 6, 2}{
      \draw[pointminus] (v\i) circle (1.5pt); }
    \fill[faceplus]
    (v1.center) -- (v5.center) -- (v7.center) -- (v3.center) -- cycle;
    \foreach \i in {1, 5, 7, 3}{
      \draw[pointplus] (v\i) circle (1.5pt); }
    \node at (0.25, 1.25, 1) {B};
  \end{scope}

  % C effect.
  \begin{scope}[xshift = 2.5cm, yshift = -2.5cm]
    \cube{};
    \fill[faceplus]
    (v4.center) -- (v5.center) -- (v7.center) -- (v6.center) -- cycle;
    \foreach \i in {4, 5, 7, 6}{
      \draw[pointplus] (v\i) circle (1.5pt); }
    \fill[faceminus]
    (v0.center) -- (v1.center) -- (v3.center) -- (v2.center) -- cycle;
    \foreach \i in {0, 1, 3, 2}{
      \draw[pointminus] (v\i) circle (1.5pt); }
    \node at (0.25, 1.25, 1) {C};
  \end{scope}

  % A:B effect.
  \begin{scope}[xshift = -2.5cm, yshift = -5cm]
    \cube{};
    \fill[faceplus]
    (v0.center) -- (v5.center) -- (v7.center) -- (v2.center) -- cycle;
    \foreach \i in {0, 5, 7, 2}{
      \draw[pointplus] (v\i) circle (1.5pt); }
    \fill[faceminus]
    (v1.center) -- (v4.center) -- (v6.center) -- (v3.center) -- cycle;
    \foreach \i in {1, 4, 6, 3}{
      \draw[pointminus] (v\i) circle (1.5pt); }
    \node at (0.25, 1.25, 1) {BC};
  \end{scope}

  % A:C effect.
  \begin{scope}[xshift = 0.0cm, yshift = -5cm]
    \cube{};
    \fill[faceminus]
    (v2.center) -- (v4.center) -- (v5.center) -- (v3.center) -- cycle;
    \foreach \i in {2, 4, 5, 3}{
      \draw[pointminus] (v\i) circle (1.5pt); }
    \fill[faceplus]
    (v0.center) -- (v6.center) -- (v7.center) -- (v1.center) -- cycle;
    \foreach \i in {0, 6, 7, 1}{
      \draw[pointplus] (v\i) circle (1.5pt); }
    \node at (0.25, 1.25, 1) {AC};
  \end{scope}

  % B:C effect.
  \begin{scope}[xshift = 2.5cm, yshift = -5cm]
    \cube{};
    \fill[faceplus]
    (v0.center) -- (v3.center) -- (v7.center) -- (v4.center) -- cycle;
    \foreach \i in {0, 3, 7, 4}{
      \draw[pointplus] (v\i) circle (1.5pt); }
    \fill[faceminus]
    (v1.center) -- (v5.center) -- (v6.center) -- (v2.center) -- cycle;
    \foreach \i in {1, 5, 6, 2}{
      \draw[pointminus] (v\i) circle (1.5pt); }
    \node at (0.25, 1.25, 1) {AB};
  \end{scope}

  \begin{scope}[xshift = 0.0cm, yshift = -7.5cm]
    \cube{};

    % \draw[blue]
    % (v0.center) -- (v5.center) -- (v6.center) -- cycle;
    % \draw[blue]
    % (v0.center) -- (v3.center) -- (v5.center) -- cycle;
    % \draw[blue]
    % (v3.center) -- (v5.center) -- (v6.center) -- cycle;
    % \draw[blue]
    % (v0.center) -- (v3.center) -- (v6.center) -- cycle;
    %
    % \draw[orange]
    % (v1.center) -- (v2.center) -- (v7.center) -- cycle;
    % \draw[orange]
    % (v1.center) -- (v2.center) -- (v4.center) -- cycle;
    % \draw[orange]
    % (v2.center) -- (v4.center) -- (v7.center) -- cycle;
    % \draw[orange]
    % (v1.center) -- (v4.center) -- (v7.center) -- cycle;

    \foreach \i in {1, 2, 7, 4}{
      \draw[pointplus] (v\i) circle (1.5pt); }
    \foreach \i in {0, 5, 6, 3}{
      \draw[pointminus] (v\i) circle (1.5pt); }
    \node at (0.25, 1.25, 1) {ABC};
  \end{scope}

\end{tikzpicture}%


\section{3d 正方形2}

\begin{tikzpicture}[%
  node distance = 4ex,
  scale = 3,
  thick,
  > = latex,
  z = {(0.45, 0.25)},
  edge/.style = {draw, thick, -, black},
  axispath/.style = {draw, ->, shorten <= 1ex, shorten >= 1ex},
  sinal/.style = {inner sep = 1pt, thin, opacity = 0.4,
    fill = blue, circle, text opacity = 1},
  ]

  \def\cube{

    \foreach \x in {0, 1, 2} {
      \foreach \y in {0, 1, 2} {
        \coordinate (v\x\y) at (\x, \y);
        % \draw[fill = black] (v\x\y\z) circle (0.8pt);
        \node[draw, circle, inner sep = 0.2ex,
          fill = white, font = \footnotesize] at (v\x\y) {\x\y};
      }
    }

    \begin{scope}[on background layer]
      \foreach \u in {0, 1, 2} {
        \draw[edge] (v0\u) -- (v2\u);
        \draw[edge] (v\u0) -- (v\u2);
      }
    \end{scope}

    % Axis text.
    \node[below of = v00, sinal, fill = blue] (Alow) {$0$};
    \node[below of = v10, sinal, fill = green, label = {[below = 1ex]-90:A}] (Amid) {$1$};
    \node[below of = v20, sinal, fill = orange] (Ahig) {$2$};
    \path[axispath] (Alow) edge (Amid) (Amid) edge (Ahig);

    \node[right of = v20, sinal, fill = blue] (Blow) {$0$};
    \node[right of = v21, sinal, fill = green, label = {[right = 1ex]0:B}] (Bmid) {$1$};
    \node[right of = v22, sinal, fill = orange] (Bhig) {$2$};
    \path[axispath] (Blow) edge (Bmid) (Bmid) edge (Bhig);

  } % \cube

  \begin{scope}[]
    \cube{};
  \end{scope}

\end{tikzpicture}%


\section{3d 球}

\newcommand\pgfmathsinandcos[3]{%
  \pgfmathsetmacro#1{sin(#3)}%
  \pgfmathsetmacro#2{cos(#3)}%
}
\newcommand\LongitudePlane[3][current plane]{%
  \pgfmathsinandcos\sinEl\cosEl{#2} % elevation
  \pgfmathsinandcos\sint\cost{#3} % azimuth
  \tikzset{#1/.estyle={cm={\cost,\sint*\sinEl,0,\cosEl,(0,0)}}}
}
\newcommand\LatitudePlane[3][current plane]{%
  \pgfmathsinandcos\sinEl\cosEl{#2} % elevation
  \pgfmathsinandcos\sint\cost{#3} % latitude
  \pgfmathsetmacro\yshift{\cosEl*\sint}
  \tikzset{#1/.estyle={cm={\cost,0,0,\cost*\sinEl,(0,\yshift)}}} %
}
\newcommand\DrawLongitudeCircle[2][1]{
  \LongitudePlane{\angEl}{#2}
  \tikzset{current plane/.prefix style={scale=#1}}
   % angle of "visibility"
  \pgfmathsetmacro\angVis{atan(sin(#2)*cos(\angEl)/sin(\angEl))} %
  \draw[current plane,thin,black] (\angVis:1) arc (\angVis:\angVis+180:1);
  \draw[current plane,thin,dashed] (\angVis-180:1) arc (\angVis-180:\angVis:1);
}%this is fake: for drawing the grid
\newcommand\DrawLongitudeCirclered[2][1]{
  \LongitudePlane{\angEl}{#2}
  \tikzset{current plane/.prefix style={scale=#1}}
   % angle of "visibility"
  \pgfmathsetmacro\angVis{atan(sin(#2)*cos(\angEl)/sin(\angEl))} %
  \draw[current plane,red,thick] (150:1) arc (150:180:1);
  %\draw[current plane,dashed] (-50:1) arc (-50:-35:1);
}%for drawing the grid
\newcommand\DLongredd[2][1]{
  \LongitudePlane{\angEl}{#2}
  \tikzset{current plane/.prefix style={scale=#1}}
   % angle of "visibility"
  \pgfmathsetmacro\angVis{atan(sin(#2)*cos(\angEl)/sin(\angEl))} %
  \draw[current plane,black,dashed, ultra thick] (150:1) arc (150:180:1);
}
\newcommand\DLatred[2][1]{
  \LatitudePlane{\angEl}{#2}
  \tikzset{current plane/.prefix style={scale=#1}}
  \pgfmathsetmacro\sinVis{sin(#2)/cos(#2)*sin(\angEl)/cos(\angEl)}
  % angle of "visibility"
  \pgfmathsetmacro\angVis{asin(min(1,max(\sinVis,-1)))}
  \draw[current plane,dashed,black,ultra thick] (-50:1) arc (-50:-35:1);

}
\newcommand\fillred[2][1]{
  \LongitudePlane{\angEl}{#2}
  \tikzset{current plane/.prefix style={scale=#1}}
   % angle of "visibility"
  \pgfmathsetmacro\angVis{atan(sin(#2)*cos(\angEl)/sin(\angEl))} %
  \draw[current plane,red,thin] (\angVis:1) arc (\angVis:\angVis+180:1);

}

\newcommand\DrawLatitudeCircle[2][1]{
  \LatitudePlane{\angEl}{#2}
  \tikzset{current plane/.prefix style={scale=#1}}
  \pgfmathsetmacro\sinVis{sin(#2)/cos(#2)*sin(\angEl)/cos(\angEl)}
  % angle of "visibility"
  \pgfmathsetmacro\angVis{asin(min(1,max(\sinVis,-1)))}
  \draw[current plane,thin,black] (\angVis:1) arc (\angVis:-\angVis-180:1);
  \draw[current plane,thin,dashed] (180-\angVis:1) arc (180-\angVis:\angVis:1);
}%Defining functions to draw limited latitude circles (for the red mesh)
\newcommand\DrawLatitudeCirclered[2][1]{
  \LatitudePlane{\angEl}{#2}
  \tikzset{current plane/.prefix style={scale=#1}}
  \pgfmathsetmacro\sinVis{sin(#2)/cos(#2)*sin(\angEl)/cos(\angEl)}
  % angle of "visibility"
  \pgfmathsetmacro\angVis{asin(min(1,max(\sinVis,-1)))}
  %\draw[current plane,red,thick] (-\angVis-50:1) arc (-\angVis-50:-\angVis-20:1);
\draw[current plane,red,thick] (-50:1) arc (-50:-35:1);

}

\tikzset{%
  >=latex,
  inner sep=0pt,%
  outer sep=2pt,%
  mark coordinate/.style={inner sep=0pt,outer sep=0pt,minimum size=3pt,
    fill=black,circle}%
}



\begin{figure}[]
	\begin{tikzpicture}[scale=1,every node/.style={minimum size=1cm}]
	%% some definitions
	
	\def\R{4} % sphere radius
	
	\def\angEl{25} % elevation angle
	\def\angAz{-100} % azimuth angle
	\def\angPhiOne{-50} % longitude of point P
	\def\angPhiTwo{-35} % longitude of point Q
	\def\angBeta{30} % latitude of point P and Q
	
	%% working planes
	
	\pgfmathsetmacro\H{\R*cos(\angEl)} % distance to north pole
	\LongitudePlane[xzplane]{\angEl}{\angAz}
	\LongitudePlane[pzplane]{\angEl}{\angPhiOne}
	\LongitudePlane[qzplane]{\angEl}{\angPhiTwo}
	\LatitudePlane[equator]{\angEl}{0}
	\fill[ball color=white!10] (0,0) circle (\R); % 3D lighting effect
	\coordinate (O) at (0,0);
	\coordinate[mark coordinate] (N) at (0,\H);
	\coordinate[mark coordinate] (S) at (0,-\H);
	\path[xzplane] (\R,0) coordinate (XE);
	
    %defining points outsided the area bounded by the sphere
	\path[qzplane] (\angBeta:\R+5.2376) coordinate (XEd);
	\path[pzplane] (\angBeta:\R) coordinate (P);%fino alla sfera
	\path[pzplane] (\angBeta:\R+5.2376) coordinate (Pd);%sfora di una quantità pari a 10 dopo la sfera
	\path[pzplane] (\angBeta:\R+5.2376) coordinate (Td);%sfora di una quantità pari a 10 dopo la sfera
	\path[pzplane] (\R,0) coordinate (PE);
    \path[pzplane] (\R+4,0) coordinate (PEd);
	\path[qzplane] (\angBeta:\R) coordinate (Q);
	\path[qzplane] (\angBeta:\R) coordinate (Qd);%sfora di una quantità pari a 10 dopo la sfera
	
	\path[qzplane] (\R,0) coordinate (QE);
    \path[qzplane] (\R+4,0) coordinate (QEd);%sfora di una quantità 10 dalla sfera sul piano equat.


    \DrawLongitudeCircle[\R]{\angPhiOne} % pzplane
    \DrawLongitudeCircle[\R]{\angPhiTwo} % qzplane
    \DrawLatitudeCircle[\R]{\angBeta}
    \DrawLatitudeCircle[\R]{0} % equator
	%labelling north and south
	\node[above=8pt] at (N) {$\mathbf{N}$};
	\node[below=8pt] at (S) {$\mathbf{S}$};
	
	\draw[-,dashed, thick] (N) -- (S);
	\draw[->] (O) -- (P);
	\draw[dashed] (XE) -- (O) -- (PE);
	\draw[dashed] (O) -- (QE);
	%connecting Points outside the sphere
	\draw[-,dashed,black,very thick] (O) -- (Pd);
	\draw[-,dashed,black,very thick] (O) -- (PEd);
    \draw[-,dashed,black,very thick] (O) -- (QEd);
    \draw[-,dashed,black,very thick] (O) -- (XEd);
    \draw[dashed] (XE) -- (O) -- (PE);
    %draw black thick flat grid
    \draw[-,ultra thick,black] (Pd) -- (PEd) node[below, left] {$P_1$};%verticale sinistro
    \draw[-,ultra thick,black] (PEd) -- (QEd)node[below, right] {$P_3$};%orizzontale inferiore
    \draw[-,ultra thick,black] (Pd)-- (XEd)node[above, right] {$P_2$};%orizzontale superiore	
    \draw[-,ultra thick,black] (XEd) -- (QEd);	
    		
	\draw[pzplane,->,thin] (0:0.5*\R) to[bend right=15]
	    node[midway,right] {$\lambda$} (\angBeta:0.5*\R);
	\path[pzplane] (0.5*\angBeta:\R) node[right] {$$};
	\path[qzplane] (0.5*\angBeta:\R) node[right] {$$};
	\draw[equator,->,thin] (\angAz:0.5*\R) to[bend right=30]
	    node[pos=0.4,above] {$\phi_1$} (\angPhiOne:0.5*\R);
	\draw[equator,->,thin] (\angAz:0.6*\R) to[bend right=35]
	    node[midway,below] {$\phi_2$} (\angPhiTwo:0.6*\R);
			\path[xzplane] (0:\R) node[below] {$$};
	\path[xzplane] (\angBeta:\R) node[below left] {$$};
    \foreach \t in {0,2,...,30} { \DrawLatitudeCirclered[\R]{\t} }
	\foreach \t in {130,133,...,145} { \DrawLongitudeCirclered[\R]{\t} }
	
	%drawing grids on the spere invoking DLongredd and DrawLongitudeCirclered
	
	\foreach \t in {130,145,...,145} { \DLongredd[\R+3]{\t} }
	\foreach \t in {130,133,...,145} { \DrawLongitudeCirclered[\R+3]{\t} }

	\foreach \t in {0,30,...,30} { \DLatred[\R+3]{\t} }
    \foreach \t in {0,2,...,30} { \DrawLatitudeCirclered[\R+3]{\t} }
	
    %labelling
    \draw[-latex,thick](4,-5.5)node[left]{$\mathsf{Grid(s)\ in\ Fig. \ (\ref{fig:Grid})}$}
    	         to[out=0,in=270] (5.8,-2.3);
    \draw[thick](3.6,-6)node[left]{$[\mathsf{Rectilinear}]$};
    	
\end{tikzpicture}
	\caption[Representation of spherical and regular computational grids used by SWAN]
    {Representation of spherical (red) and cartesian (black) co-ordinate systems. Latter 
    gives an example of unstructured grids. Both unstructured. Conversion from former 
    to latter involves a deformation factor which is acceptable within a given spatial limit. 
    For my case, only unstructured flat meshes are employed (\textit{Lisboa} Geodetic 
    datum: black grid on the right). Confront above represented points ($P_1,P_2,P_3$) with 
    Fig.(\ref{fig:Grid}). \\Mathematically frames change accordingly: see Eq.(\ref{eq:actbal2sph}).}
	\label{fig:frames}
      \end{figure}


\section{layer}


      \begin{tikzpicture}
  \tikzset{
   block/.style={
    draw=red,
    fill=white,
    minimum size=1.5cm,
    rounded corners,
    align=center,
   },
  }
  \node[block] (2D) {$2$D\\Histogramm};
  \node[block, above left=-.5cm and 1cm of 2D] (E) {E\\signal};
  \node[block, below left=-.5cm and 1cm of 2D] (DE) {$\Delta$E\\signal};
  \node[block, right=of 2D] (region) {$p$ signal\\selection};
  \node[block, right=of region] (projection) {E detector\\projection};
  \node[block, right=of projection] (smoothing) {smoothing};
  \node[block, right=of smoothing] (rebinning) {rebinning};
  \begin{scope}[on background layer]
   \node [fit=(E) (DE) (rebinning), fill= gray!30, rounded corners, inner sep=.5cm, label={[red]below:Your label text here}] {};
  \end{scope}
 \end{tikzpicture}


\end{document}