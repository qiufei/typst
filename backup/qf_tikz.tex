
\documentclass{standalone}
\usepackage[UTF8,punct]{ctex}

\usepackage{graphicx}
\usepackage{color}
\usepackage{xcolor}
\definecolor{darkgreen}{rgb}{0.13,0.53,0.53}
\definecolor{darkblue}{HTML}{224271}

\usepackage{amsfonts, amsmath} 

\usepackage{ifthen}
\usepackage{bm}

\usepackage{pgfplots}
\usepackage{pgfplotstable}
\usepackage{pgfcalendar}
\pgfplotsset{compat=newest}
\usepgfplotslibrary{groupplots}
\usepgfplotslibrary{fillbetween}

\usepackage{tikz}
\usetikzlibrary{
  arrows,
  automata,
  backgrounds,
  calc,
  calendar,
  chains,
  decorations,
  decorations.pathreplacing,
  decorations.pathmorphing,
  decorations.markings,
  decorations.text,
  er,
  external,
  fadings,
  fit,
  fixedpointarithmetic,
  fpu,
  folding,
  intersections,
  matrix,
  mindmap,
  lindenmayersystems,
  positioning,
  patterns,
  petri,
  plotmarks,
  shapes,
  shadows,
  shadings,
  snakes,
  spy,
  topaths,
  trees,
  turtle,
  through}

%%%%%%%%%%%%%%%%%%%%%


\begin{document}

\begin{tikzpicture}[
  declare function={mu1=1;},
  declare function={mu2=2;},
  declare function={sigma1=0.45;},
  declare function={sigma2=0.6;},
  declare function={
    normal(\m,\s)=1/(2*\s*sqrt(pi))*exp(-(x-\m)^2/(2*\s^2));
  },
  declare function={
    bivar(\ma,\sa,\mb,\sb)=
    1/(2*pi*\sa*\sb)*exp(-((x-\ma)^2/\sa^2+(y-\mb)^2/\sb^2))/2;}
]

%系统性金融风险的联合分布


\pgfplotsset{width=11cm}
\pgfplotsset{%
  colormap={whitered}{color(0cm)=(white);
    color(1cm)=(darkblue)} %color(1cm)=(orange!75!red)
}
  

  \begin{axis}[
    colormap name=whitered,
    width=11cm,
    view={45}{45},
    enlargelimits=false,
    grid=major,
    domain=0:2,
    y domain=0.5:3.5,
    samples=26,
    xlabel=$x$,
    ylabel=$y$,    
    zlabel={$P$},
    xticklabels=\empty,
    yticklabels=\empty,
    zticklabels=\empty,
    axis line style={draw=none},
    tick style={draw=none},
    scale = 1.6,
    legend style={at={(0.45,-0.08)},anchor=south,draw=none} % draw=none不要legend的外边框
    ]
    %   x,y的联合分布
    \addplot3 [surf] {bivar(mu1, sigma1, mu2, sigma2)};
    
    % x的边缘分布
    \addplot3 [domain=0:2, samples=31, samples y=0, thick, smooth,color = pink!70!red]
    (x, 3.5, {normal(mu1, sigma1)})
    node[midway,below,yshift = -2cm,text width = 5cm,align = center] {风险负担分布:\\ 边缘概率分布};
    % 这里第二个参数即y=3,是因为y值的范围终点是3
    
    
    % y的边缘分布
    \addplot3 [domain=0.5:3.5, samples=31, samples y=0, thick, smooth,color = darkgreen]
    (0, x, {normal(mu2, sigma2)})
    node[midway,below,yshift = -2cm,text width = 5cm,align =center] {风险能力分布:\\ 边缘概率分布};
    % 这里第一个参数为常数,代表在x=c的面上画图,这里c=0是因为前面设定x值的范围是从0开始,domain=0:2
    % 第二个参数变动,就是y值的范围,前面设定的y的范围是1:3
    
    \legend{系统性金融风险的联合概率分布}


    
    % 
    % \node[above] at (axis cs:0,1.5,0.28)  [pin=-15:风险能力分布$P(x_1)$] {边缘分布};
    % y 边际
    % \node at (axis cs:0.7,3,0.32) [pin=-15:风险负担分布$P(x_2)$] {};

    
  \end{axis}
\end{tikzpicture}


\end{document}