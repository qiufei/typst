
\documentclass[a4paper]{article}
\usepackage[UTF8,punct]{ctex}

\usepackage[]{graphicx}
\usepackage[]{color}

\usepackage[brazil]{babel}
\usepackage[utf8]{inputenc}

% \usepackage{palatino}
% \usepackage{eulervm}
\usepackage[condensed, math]{iwona}
% \usepackage[scaled=0.8]{beramono}
\usepackage{inconsolata}


\usepackage[T1]{fontenc}


\usepackage{scrextend}
% \changefontsizes[11pt]{8pt}
\usepackage{array}
\usepackage{colortbl}
\usepackage{booktabs}
\usepackage{filecontents}
\usepackage{multirow}
\usepackage{eurosym}
\usepackage{amsfonts, amssymb, amsxtra, amsmath, amsbsy} 

\usepackage{xcolor}
\usepackage{listings}
\usepackage{ifthen}

\usepackage{tikz,bm}
\usepackage{pgfplots}

\usepackage{pgfplotstable}
\usepackage{pgfcalendar}
\pgfplotsset{compat=newest}
\usepgfplotslibrary{groupplots}
\usepgfplotslibrary{fillbetween} 
\usetikzlibrary{
  arrows,
  through,
  positioning,
  matrix,
  calc,
  decorations.pathreplacing,
  decorations.pathmorphing,
  decorations.markings,
  decorations.text,
  shapes,
  backgrounds,
  shadows,
  trees,
  fit,
  snakes,
  patterns,
  mindmap,
  intersections,
  calendar,
  plotmarks,
  spy}



\pagestyle{empty}


\definecolor{darkgreen}{rgb}{0.13,0.53,0.53}




\begin{document}


\section{相交的圆}

\begin{tikzpicture}[scale=1.2]
\coordinate [label=left:$A$] (A) at (0,0);
\coordinate [label=right:$B$] (B) at (1.25,0.25);
\draw (A) -- (B);
\node (D) [draw,circle through=(B),label=left:$D$] at (A) {}; \node (E) [draw,circle through=(A),label=right:$E$] at (B) {};
\coordinate[label=above:$C$] (C) at (intersection 2 of D and E); \draw [red] (A) -- (C);
\draw [red] (B) -- (C);
\end{tikzpicture}


\section{填充}

\begin{tikzpicture}
        \begin{axis}[very thick,
                     samples = 100,
                     xlabel = H,
                     ylabel = B,
                     xmin = -7,
                     xmax = 7,
                     ymin = -4,
                     ymax = 4,
                     axis x line = middle,
                     axis y line = middle,
                     ticks = none]
            \addplot[dashed] plot (\x, 2.5);
            \addplot[dashed] plot (\x,-2.5);
            \addplot[red, name path=A] plot (\x, {5/(1 + exp(-1.7*\x+1.5))-2.5});
            \addplot[red, name path=B] plot (\x, {5/(1 + exp(-1.7*\x-1.5))-2.5});
            \addplot[red!20] fill between[of=A and B];
        \end{axis}
    \end{tikzpicture}


\section{轨迹上的不同分布}

\begin{tikzpicture}[domain=-0.25:9, scale=0.7, >=latex]
  % \draw[very thin,color=gray!30] (-1.1,-1.1) grid (12.1,7.1);
  \draw[->] (-0.2,0) -- (9.2,0) node[right] {$x$};
  \draw[->] (0,-0.2) -- (0,5.5) node[above] {$Y$};

  \begin{scope}[line width=1pt]
    \coordinate (a1) at (0,0);
    \coordinate (a2) at (4,3);
    \coordinate (a3) at (9,5);
    \draw (a1) ..controls (2,1) and (1,5).. (a2);
    \draw (a2) ..controls (7,1) and (7,3).. (a3);
    \node[right] at (a3) {$\eta(x)$};
  \end{scope}

  \begin{scope}[
    color=gray, fill opacity=0.3, fill=gray, smooth, domain=-1:1]

    \def\x{1.5}; \def\y{2.4};
    \filldraw[xshift=\x cm, yshift=\y cm]
      (0,-0.8) -| (0.8,0) |- (0,0.9);
    \draw (\x,\y) -- ++(0.8,0);
    \draw[dashed] (\x,0) -- ++(0,5.5);

    \def\x{3.75}; \def\y{3.15};
    \filldraw[xshift=\x cm, yshift=\y cm]
      (0,-1) -- (1,0) -- (0,1);
    \draw (\x,\y) -- ++(1,0);
    \draw[dashed] (\x,0) -- ++(0,5.5);

    \def\x{6}; \def\y{2.7};
    \filldraw[xshift=\x cm, yshift=\y cm]
      plot[id=x, rotate=-90]
      function{0.55*(1+x)**3*(1-x)**0.9};
    \draw (\x,2.17) -- ++(1,0);
    \draw[dashed] (\x,0) -- ++(0,5.3);

    \def\x{8}; \def\y{3.85};
    \filldraw[xshift=\x cm, yshift=\y cm]
      plot[id=x, rotate=-90, domain=-1.2:1.2]
      function{exp(-(x)**2/0.29)};
    \draw (\x,\y) -- ++(1,0);
    \draw[dashed] (\x,0) -- ++(0,5.5);

  \end{scope}

  \coordinate (a) at (8.25,3.5);
  \node (b) at (10,2.8) {$[Y|x]$};
  \draw (a) edge[out=0, in=180,->] (b);

\end{tikzpicture}


\section{轨迹分布}

\begin{tikzpicture}[
  declare function={
    normal(\m,\s)=
    1/(2*\s*sqrt(pi))*exp(-(x-\m)^2/(2*\s^2));
  },
  declare function={
    reg(\x,\a,\b)=\a+\b*\x;
  },
  declare function={
    qua(\x,\a,\b,\c)=\a+\b*\x+\c*\x^2;
  },
  >=stealth,
  cx/.style={fill=white, font=\footnotesize},
  pth/.style={draw, ->, color=darkgreen}]
  \begin{axis}[
    width=9cm, height=6cm,
    xlabel=$x$: preditora,
    ylabel=$y$: resposta,
    xtick=\empty,
    xticklabels=\empty,
    yticklabels=\empty,
    extra x ticks={-0.9,-0.4,0.1,0.6},
    extra x tick labels={$x_1$,$x_2$,$x_3$,$x_4$},
    extra tick style={grid=major}]
    \addplot[color=black, thick, domain=-1:1, samples=2]
      (x, {reg(x,-0.75,3.5)});
    \addplot[color=black, thick, domain=-1:1,
      samples=20, color=darkgreen]
      (x, {qua(x,0,3,-2)});
    \draw[draw=red] (axis cs:9,9) -- node[pos=0.65, above, sloped]
      {\footnotesize $E(y) = \beta_0+\beta_1 x$} ++(axis cs:1.5,1.5);
    \pgfplotsinvokeforeach{-0.9,-0.4,0.1,0.6}{
      \addplot[domain=-2:2, samples=30,
        fill=darkgreen, fill opacity=0.25, draw=none]
        ({#1+0.5*normal(0,0.5)}, {x+qua(#1,0,3,-2)});
      \draw[dashed] (axis cs:#1, {qua(#1,0,3,-2)}) -- 
        (axis cs:#1+0.28, {qua(#1,0,3,-2)});
    }
  \end{axis}
\end{tikzpicture}


\section{轨迹分布2}
\label{sec:3}

\begin{tikzpicture}[domain=0:8, scale=0.7]
  % \draw[very thin,color=gray!30] (-1.1,-1.1) grid (12.1,7.1);
  \draw[->] (-0.2,0) -- (9.2,0) node[right] {$x$};
  \draw[->] (0,-0.2) -- (0,5.5) node[above] {$E(Y|x)$};
  \draw[color=darkgreen, thick]
      plot[id=x] function{0+0.7*x}
      node[right] {$\mu = \beta_0+\beta_1\cdot x$};
  \foreach \x/\y in {1.5/1.05, 3/2.1, 4.5/3.15, 6/4.2}{
    \begin{scope}[
      xshift=\x cm, yshift=2cm +\y cm,
      rotate=-90, smooth, domain=0.5:3.5]
      \draw[color=darkgreen] (1,0) -- (3,0);
      \draw[color=darkgreen!70!blue] (2,0) -- (2,1);
      \filldraw[
        color=darkgreen!70!black, fill opacity=0.3, fill=darkgreen!70!black]
        plot[id=x] function{exp(-(x-2)**2/0.2)};
    \end{scope}
  }
  \path[->, color=darkgreen!70!black, thick] (6.5,3.7) edge[bend right=45] 
    node[at end, right]
    {$\frac{1}{\sqrt{2\pi\sigma^2}}
      \exp\{-\frac{(y-\mu)^2}{2\sigma^2}\}$}
    +(1,-1);
\end{tikzpicture}


\section{3d轨迹分布}
\label{sec:3d-1}

\begin{tikzpicture}[
  >=stealth,
  declare function={
    normal(\m,\s)=1/(2*\s*sqrt(pi))*exp(-(x-\m)^2/(2*\s^2));
  },
  declare function={
    reg(\x,\a,\b)=\a+\b*\x;
  }]

  \begin{axis}[
    width=9cm,
    height=7cm,
    view={45}{50},
    xlabel=$x$: preditora,
    ylabel=$y$: resposta,
    zlabel=densidade,
    xlabel style={rotate=-25},
    ylabel style={rotate=25},
    zlabel style={rotate=0},
    samples=60,
    domain=-4:12,
    % title=Regress\~{a}o linear simples,
    % grid=both,
    % minor tick num=2,
    % xtick={1,3,5,7},
    % xticklabel=\empty,
    xtick=\empty,
    ztick=\empty,
    ytick=\empty,
    samples y=0,
    zmin=0,
    area plot/.style={
      fill opacity=0.5,
      draw=none,
      fill=darkgreen!40,
      mark=none,
      smooth
    }]

    \addplot3[smooth, mark=none, color=black, thick,
      domain=0:8, samples=10] (x,{reg(x,0,1)},0);
    \pgfplotsinvokeforeach{1,3,...,7}{
      \addplot3 [area plot] (#1,x,{normal(#1,1)});
      \draw [dashed] (axis cs:#1,#1,0) -- (axis cs:#1,#1,0.28);
      \draw [gray, dashed] (axis cs:#1,-4,0) -- (axis cs:#1,12,0);
    }
    \draw[<-, shorten <=2pt, shorten >=0pt]
      (axis cs:6,6,0) -- (axis cs:6,6,0.3)
      node[above] {$E(Y)=X\beta$};
  \end{axis}
\end{tikzpicture}


\section{轨迹分布图1}

\begin{tikzpicture}
  [domain=-0.25:9, xscale=0.8, yscale=5,
   aponta/.style={
    color=darkgreen!50!blue, rounded corners=5pt, -latex, thick
  }]
  \draw[-latex] (-0.2,0) -- (10,0) node[right] {$x$};
  \draw[-latex] (0,-0.01) -- (0,1) node[above] (E) {$\textrm{E}(Y|x)$};
  \def\A{-3}; \def\B{0.8};
  \draw[color=black, thick]
  plot[id=x] function{1/(1+exp(-\A-\B*x))}
  node[right] (eta)
    {$\displaystyle\frac{1}{1+\exp\{-(\theta_0+\theta_1 x)\}}$};
  \node (Q) at (5,25/20) {$Q(Y|x) = \eta(x, \theta)$};
  \node (N) at (5,22/20) {$[Y|x]\sim$ Beta($\mu$, $\phi$)};
  \path[aponta] (3.9,12/20) edge[bend left=10] (N);
  \draw[aponta] (Q) -| (E);
  \draw[aponta] (Q) -| (eta);
  \def\parphi{40}; 
  \def\parmu{0.5}; \def\factor{0.14}
  \foreach[evaluate=\x as \parmu using 1/(1+exp(-\A-\B*\x))]
    \x in {1, 2.4, ..., 7}{
      \draw[color=gray, dashed] (\x,0) -- ++(0,1);
      \begin{scope}[xshift=\x cm, rotate=90, yscale=1]
        % \draw (\parmu,0) -- ++(0,-0.5);
        \filldraw[fill opacity=0.3, fill=gray!70!black]
          plot[domain=(\parmu-0.25):(\parmu+0.25), smooth, samples=100]
          function {-\factor*gamma(\parphi)/(gamma(\parphi*\parmu)*gamma((1-\parmu)*\parphi))*x**(\parmu*\parphi-1)*(1-x)**((1-\parmu)*\parphi-1)};
      \end{scope}
    }

\end{tikzpicture}%


\section{ 轨迹分布图2}
\label{sec:2-1}

\begin{tikzpicture}[%
  domain=-0.25:9, xscale=0.8, yscale=0.25,
  aponta/.style={
    color=darkgreen!50!blue, rounded corners=5pt, -latex, thick
  }]

  \draw[-latex] (-0.2,0) -- (10,0) node[right] {$x$};
  \draw[-latex] (0,-0.2) -- (0,20) node[above] (E) {$\textrm{E}(Y|x)$};

  \def\A{-3}; \def\B{0.8};
  \draw[color=black, thick]
    plot[id=x] function{20/(1+exp(-\A-\B*x))}
    node[right] (eta)
    {$\displaystyle\frac{n}{1+\exp\{-(\theta_0+\theta_1 x)\}}$};

  \node (Q) at (5,25) {$Q(Y|x) = \eta(x, \theta)$};
  \node (N) at (5,22) {$[Y|x]\sim$ Binomial($p$, $n$)};
  \path[aponta] (5.8,14.5) edge[bend left=10] (N);

  \draw[aponta] (Q) -| (E);
  \draw[aponta] (Q) -| (eta);
  
  \foreach[evaluate=\x as \p using 1/(1+exp(-\A-\B*\x))]
    \x in {0.5, 2.2, ..., 8.5}{
      \draw[color=gray, dashed] (\x,0) -- ++(0,20);
      \def\lim{20};
      % \node[below] at (\x,0) {\p};
      \begin{scope}[xshift=\x cm, rotate=90, yscale=4]
        \draw[color=black, very thick]
        plot[ycomb, samples=\lim+1, domain=0:\lim]
        function {-(gamma(\lim+1)/(gamma(x+1)*
          (gamma(\lim-x+1))))*\p**x*(1-\p)**(\lim-x)};
      \end{scope}
    }
    
\end{tikzpicture}




\section{test}
\label{sec:test}

\begin{tikzpicture}[
  domain=0:9, xscale=0.8, yscale=0.25,
  aponta/.style={
    color=darkgreen!50!blue, rounded corners=5pt, -latex, thick}
  ]
  \draw[-latex] (-0.2,0) -- (10,0) node[right] {$x$};
  \draw[-latex] (0,-0.2) -- (0,20)
    node[above] (E) {$\textrm{E}(Y|x,a_i)$};
  \def\A{19}; \def\V{2};
  \draw[color=black, thick, domain=0:11]
    plot[id=x] function{\A*x/(\V+x)}
    node[right] (eta) {$\displaystyle\frac{\theta_a x}{\theta_v+x}$};
  \node (Q) at (5,25) {$Q(Y|x,a_i) = \eta(x, \theta, a_i)$};
  \node (N) at (5,22) {$[Y|x,a_i]\sim$ Normal($\mu$,$\sigma$)};
  \path[aponta] (2.35,12) edge[bend left=10] (N);
  \draw[aponta] (Q) -| (E);
  \draw[aponta] (Q) -| (eta);
  \draw[color=black, very thin, domain=0:11]
    plot[id=x, samples=100] function{(\A-5.5)*x/(\V+x)}
    node[right] {$\displaystyle\frac{(\theta_a+a_i)\, x}{\theta_v+x}$};
  \foreach \Aa in {-5.5,-2.7,-0.5,0,1.2,1.8,4.5}{
    \draw[color=black, very thin]
      plot[id=x, samples=100] function{(\A+\Aa)*x/(\V+x)};
    \foreach[evaluate=\x as \y using (\A+\Aa)*\x/(\V+\x)]
      \x in {0.5, 2.2, ..., 8.5}{
        \draw[color=gray, dashed] (\x,0) -- ++(0,20);
        \def\sr{2.5}
        \begin{scope}[
          xshift=\x cm, yshift=\y cm,
          rotate=-90, smooth, domain=-\sr:\sr]
          \filldraw[fill opacity=0.3, fill=white!85!black]
            plot[id=x] function{0.5*exp(-(x)**2/0.5)};
      \end{scope}
    }
  }
  \def\sr{8}
  \begin{scope}[
    xshift=9.1 cm, yshift=0.83*\A cm,
    rotate=-90, smooth, domain=-\sr:\sr]
    \filldraw[fill opacity=0.3, fill=gray!50!black]
      plot[id=x] function{1.5*exp(-(x)**2/10)};
    \node (a) at (10,1.5) {$[a_i]\sim$ Normal($0$,$\sigma_a$)};
    \draw[aponta] (1,0.5) |- (a);
  \end{scope}

\end{tikzpicture}



\begin{tikzpicture}[
  domain=0:9, xscale=0.8, yscale=0.25,
  aponta/.style={
    color=darkgreen!50!blue, rounded corners=5pt, -latex, thick
  }]

  \draw[-latex] (-0.2,0) -- (10,0) node[right] {$x$};
  \draw[-latex] (0,-0.2) -- (0,20) node[above] (E) {$\textrm{E}(Y|x)$};
  \def\A{18}; \def\V{2};
  \draw[color=black, thick]
    plot[id=x] function{\A*x/(\V+x)}
    node[right] (eta) {$\displaystyle\frac{\theta_a x}{\theta_v+x}$};
  \node (Q) at (5,25) {$Q(Y|x) = \eta(x, \theta)$};
  \node (N) at (5,22) {$[Y|x]\sim$ Normal($\mu$,$\sigma$)};
  \node[below right of=E, anchor=west] (V) {$\textrm{V}(Y|x)$};
  \path[aponta] (3.2,11.8) edge[bend left=10] (N);
  \draw[aponta] (Q) -| (E);
  \draw[aponta] (Q) -| (V);
  \draw[aponta] (Q) -| (eta);
  \foreach[
    evaluate=\x as \y using \A*\x/(\V+\x),
    evaluate=\y as \s using (\y^1.25)/20]
    \x in {0.8, 3, ..., 8}{
      \draw[color=gray, dashed] (\x,0) -- ++(0,20);
      \def\factor{2.3}
      \begin{scope}[
        xshift=\x cm, yshift=\y cm,
        rotate=-90, smooth, domain=(-4*\s):(4*\s)]
        \draw[color=gray] (-2*\s,0) -- (2*\s,0);
        \filldraw[fill opacity=0.3, fill=gray!70!black]
          plot[id=x]
          function{\factor*(2*3.14)**(-0.5)*(1/\s)*
            exp(-(x)**2/(2*\s**2))};
    \end{scope}
  }
\end{tikzpicture}


\section{不同分布}


\def\tr{0.1}
\def\ts{0.6}
\def\al{1.3}
\def\n{1.6}
\def\I{0.3506} % -log(al)+log(m)/n = -log(al)+log(1-1/n)/n
\def\ti{0.4071}
\def\S{-0.1340}
\def\Sangle{-42}
\def\f1{0.8}

\begin{tikzpicture}[domain=-3:5, xscale=1.1, yscale=6, >=latex]

  \draw[very thin,color=gray!30] (-3,0)
    grid[xstep=0.5, ystep=0.1] (5,0.7);
  \draw[->, line width=1pt] (-3,0) -- (5.25,0)
    node[below] {$\log(\psi)$};
  \draw[->, line width=1pt] (-3,0) -- (-3,0.75)
    node[left] {$U(\psi)$};
  \draw[color=darkgreen!30!black, thick, smooth]
    plot[id=x, domain=-3:5] 
    function{\tr+(\ts-\tr)/(1+(\al*exp(x))**\n)**(1-1/\n)};
  \node[left] (tr) at (-3,\tr) {$U_r$};
  \node[left] (ts) at (-3,\ts) {$U_s$};
  \draw[dashed] (\I,0) node[below] {$I$} -- (\I,\ti) -- (-3,\ti)
    node[left] {$U_i$};
  \draw[color=darkgreen!30!black, dashed]
    plot[id=x, domain=-1.5:3]
    function{\ti+\S*(x-\I)};
  \draw [decorate,decoration={brace,amplitude=4pt}]
    (-3.5,\tr) -- (-3.5,\ts) node [black,midway,left] {$\Delta$};
  \draw[|<->|] (\I,\ti)++(1,0) arc (0:\Sangle:0.8 and 0.15);
  \path (\I,\ti)++(0.5*1.5\Sangle/8:1)
    node[right=-1pt] {$\tan^{-1}(S)$};
  \node[anchor=base] (vg) at (1,0.8)
    {$U(\psi) = U_r+\displaystyle\frac{U_s-U_r}{(1+(\alpha\psi)^n)^m}$};

  \begin{scope}[yshift=-0.3cm,domain=-3:5, >=latex, yscale=3]
    \draw[very thin,color=gray!30] (-3,-0.15)
      grid[xstep=0.5, ystep=0.03] (5,0);
    \draw[->, line width=1pt] (-3,-0.15) -- (5.25,-0.15)
      node[below] {$x=\log(\psi)$};
    \draw[->, line width=1pt] (-3,-0.15) -- (-3,0.03)
      node[left] {$U'(x)$};
    \draw[color=darkgreen!30!black, thick, smooth]
      plot[id=x, domain=-3:5]
      function{-(\ts-\tr)*\n*(1-1/\n)*\al**\n*exp(\n*x)*(1+(\al*exp(x))**\n)**(-1+1/\n-1)} node[right] {};
    \node[left] (z) at (-3,0) {$0$};
    \draw[dashed] (\I,0) -- (\I,\S) -- (-3,\S) node[left] {$S$};
    \node[anchor=base] (dvg) at (1,0.02)
      {$U'(x) = -(U_s-U_r) \,n m\alpha^n
        \exp\{nx\}(1+(\alpha\exp\{x\})^n)^{-m-1}$};
  \end{scope}

  \begin{scope}[yshift=-1.6cm,domain=-3:5, >=latex, yscale=1.75]
    \draw[very thin,color=gray!30] (-3,0)
      grid[xstep=0.5, ystep=0.05] (5,0.35);
    \draw[->, line width=1pt] (-3,0) -- (5.25,0)
      node[below] {$x=\log(\psi)$};
    \draw[->, line width=1pt] (-3,-0.05) -- (5.25,-0.05)
      node[below] {$r=2\sigma\cos(\delta)/(\rho g \psi)$};
    \draw[->, line width=1pt] (-3,0) -- (-3,0.38) node[left] {$f(x)$};
    \draw[color=darkgreen,thick, smooth]
    plot[id=x, domain=-3:5]
    function{\n*(1-1/\n)*\al**\n*exp(\n*x)*(1+(\al*exp(x))**\n)**(-1+1/\n-1)};
    \node[left] (z) at (-3,0) {$0$};
    \node[anchor=base] (inte) at (1,0.32)
      {$\displaystyle\int_{-\infty}^{\infty} nm\alpha^n
        \exp\{nx\}(1+(\alpha\exp\{x\})^n)^{-m-1}\,\, dx=1$};
    \draw[dashed] (\I,0) -- (\I,-2*\S);
  \end{scope}

  \begin{scope}[yshift=-1.8cm, xshift=-0.5cm, xscale=8, yscale=1.5]
    \foreach \r/\x in {10/-2, 8/0, 6/2, 4/4, 2/5.5} {
      \draw (\x/8,0) circle (\r/150);
      \draw (\x/8,0) -- ++(\r/150,0);
    }
    \node[below right] (r) at (-2/8,0) {$r$};
    \node[anchor=base] (d) at (0.2,-0.12) {tamanho de poros ($r$)};
  \end{scope}
\end{tikzpicture}



\section{两个分布}

\pgfplotsset{
  myplot/.style={
    width=12cm, height=6cm,
    xlabel=$x$, ylabel=$f(x)$,
    samples=50,
    xlabel style={at={(1,0)}, anchor=north west},
    ylabel style={rotate=-90, at={(0,1)}, anchor=south east},
    legend style={draw=none, fill=none},
  }
}

\begin{tikzpicture}[>=stealth,
  every node/.style={rounded corners},
  declare function={
    normalpdf(\x,\mu,\sigma)=
    (2*3.1415*\sigma^2)^(-0.5)*exp(-(\x-\mu)^2/(2*\sigma^2));
  }]

  \begin{axis}[myplot]
    \addplot[smooth, thick, domain=1.3:2.1]
    {normalpdf(x,1.7,0.1)};
    \addlegendentry[align=center]{$\sigma^2=0.1^2$}

    \addplot[smooth, thick, domain=1.3:2.1, color=darkgreen]
    {normalpdf(x,1.7,0.0276)};
    \addlegendentry[align=center]{$\sigma^2_{\bar{X}}=\frac{0.1^2}{10}$}

    \node[anchor=north west] at (axis description cs: 0.02, 0.95)
    {$X$: altura (m), $\mu=1.7$.};
    \node[anchor=north west, text width=3.5cm]
    at (axis description cs: 0.02, 0.83)
    {$\bar{X}$: altura m\'{e}dia (m), $n=10$.};
  \end{axis}

 
\end{tikzpicture}


\section{两个分布2}
\label{sec:2}

\def\xs{1}
\def\ys{10}

\begin{tikzpicture}[xscale=\xs, yscale=\ys, >=latex]
  % definicoes dos valores dos parametros e outras quantidades
  \def\al{1.3}
  \def\all{0.5}
  \def\n{1.6}
  \def\nn{1.9}

  % grid, eixos e anotacoes
  \draw[very thin,color=gray!30] (-3-0.2/\xs,0-0.2/\ys)
    grid[xstep=0.5, ystep=0.05] (5+0.2/\xs,0.4+0.2/\ys);
  \draw[->, line width=1pt] (-3,0) -- (5.25,0) node[below] {$r$};
  \draw[->, line width=1pt] (-3,0) -- (-3,0.43) node[left] {$f(r)$};
  \node[left] at (-3,0) {$0$};

  % funcoes
  \draw[color=darkgreen, thick, smooth] plot[id=x, domain=-3:5]
    function{\n*(1-1/\n)*\al**\n*exp(\n*x)*
      (1+(\al*exp(x))**\n)**(-1+1/\n-1)};
  \draw[color=red, thick, smooth] plot[id=x, domain=-3:5]
    function{\nn*(1-1/\nn)*\all**\nn*exp(\nn*x)*
      (1+(\all*exp(x))**\nn)**(-1+1/\nn-1)};

  % textos
  \path[->, draw] (-0.5,0.2) to[out=180, in=0] ++(-0.75,0.05)
    node[anchor=east, text ragged left, text width=9ex,
    fill=white, inner sep=0pt]
    {antes do manejo};
  \path[->, draw] (2,0.25) to[out=0, in=180] ++(0.75,0.05)
    node[anchor=west, text ragged, text width=8ex,
    fill=white, inner sep=0pt]
    {ap\'os o manejo};
\end{tikzpicture}  








\section{3d 不同分布}
\label{sec:3d-2}

\begin{tikzpicture}
  \pgfplotstableread{
    plot1   plot2   plot3   plot4
    0       0       0       0
    3.466   2.058   0       0
    4.262   2.976   0.001   0
    3.822   3.168   0.006   0.008
    2.953   2.936   0.019   0.063
    2.065   2.492   0.046   0.265
    1.332   1.977   0.092   0.734
    0.797   1.478   0.164   1.508
    0.443   1.045   0.268   2.44
    0.228   0.698   0.412   3.219
    0.107   0.438   0.598   3.524
    0.046   0.256   0.831   3.219
    0.017   0.138   1.109   2.44
    0.006   0.067   1.429   1.508
    0.002   0.029   1.78    0.734
    0       0.01    2.141   0.265
    0       0.003   2.479   0.063
    0       0.001   2.736   0.008
    0       0       2.808   0
    0       0       2.465   0
    0       0       0       0
  }\dummydata

  \begin{axis}[
    samples=30,
    domain=-4:4,
    samples y=0, ytick={1,...,4},
    zmin=0,
    area plot/.style={
      fill opacity=0.75,
      draw=orange!80!black,thick,
      fill=orange,
      mark=none,
    }]

    \pgfplotsinvokeforeach{4,3,...,1}{
      \addplot3 [area plot]
        table [x expr=\coordindex, y expr=#1, z=plot#1]
        {\dummydata};
    }
  \end{axis}
\end{tikzpicture}

\begin{tikzpicture}[
  declare function={
    normal(\m,\s)=1/(2*\s*sqrt(pi))*exp(-(x-\m)^2/(2*\s^2));
  }]

  \begin{axis}[
    samples=30,
    domain=-4:4,
    samples y=0, ytick=data,
    zmin=0,
    area plot/.style={
      fill opacity=0.75,
      draw=none,
      fill=blue!70,
      mark=none,
      smooth
    }]

    \addplot3 [black, thick] table {
      0 4 0
      -0.75 3 0
      -1.9 2 0
      -1.2 1 0
    };

    \addplot3 [area plot] (x,4,{normal(0,1)});
    \addplot3 [area plot] (x,3,{normal(-0.75,1)}) -- (axis cs:-4,3,0);
    \addplot3 [area plot] (x,2,{normal(-1.9,0.7)}) -- (axis cs:-4,2,0);
    \addplot3 [area plot] (x,1,{normal(-1.2,1.2)}) -- (axis cs:-4,1,0);
  \end{axis}
\end{tikzpicture}




\section{两个分布,不同颜色}

\def\zleft{-1.64}
\def\zright{1.64}
\def\muzero{0}
\def\muone{1}

\pgfplotsset{
  myplot/.style={
    width=14cm, height=6cm,
    % xlabel=$z$, ylabel=$f(z)$,
    samples=50,
    xmin=-6, xmax=7,
    ymax=0.5,
    xlabel style={at={(1,0)}, anchor=west},
    ylabel style={rotate=-90, at={(0,1)}, anchor=south west},
    legend style={draw=none, fill=none},
    xticklabels=\empty,
    yticklabels=\empty,
    legend pos=north west,
    legend cell align=left,
    clip=false, % Para anotar fora da area grafica.
    % extra tick style={grid=major}
  }
}

\begin{tikzpicture}[
  >=stealth,
  every node/.style={rounded corners},
  Red/.style={draw=none, text opacity=1, fill=red!70!blue, fill
    opacity=0.75},
  Yellow/.style={draw=none, text opacity=1, fill=yellow, fill
    opacity=0.25},
  Blue/.style={draw=none, text opacity=1, fill=blue, fill
    opacity=0.25},
  declare function={
    normalpdf(\x,\mu,\sigma) =
    (2*3.1415*\sigma^2)^(-0.5)*exp(-(\x-\mu)^2/(2*\sigma^2));
  }]

  \begin{axis}[
    myplot,
    extra x ticks={\muzero,\muone},
    extra x tick labels={$\mu_0$,$\mu$}]
    % Legenda.
    \addlegendentry{第一种错误 ($\alpha$)}
    \addlegendimage{only marks, mark=*, fill=red!70!blue}
    \addlegendentry{第二种错误 II ($\beta$)}
    \addlegendimage{only marks, mark=*, fill=blue, fill opacity=0.25}
    % Regioes preenchidas.
    \addplot[Red, smooth, domain=-5:\zleft]
      {normalpdf(x,\muzero,1)} \closedcycle;
    \addplot[Red, smooth, domain=\zright:5]
      {normalpdf(x,\muzero,1)} \closedcycle;
    \addplot[Yellow, smooth, domain=\zleft:\zright]
      {normalpdf(x,\muzero,1)} \closedcycle;
    \addplot[Blue, smooth, domain=\zleft:\zright]
      {normalpdf(x,\muone,1)} \closedcycle;
    % Curvas.
    \addplot[smooth, domain=-5:5] {normalpdf(x,\muzero,1)};
    \addplot[smooth, dashed, domain=-5:5] {normalpdf(x,\muone,1)};
    % Verticais das medias.
    \addplot[ycomb, mark=none, samples at={\muzero}]
      {normalpdf(x,\muzero,1)};
    \addplot[ycomb, mark=none, samples at={\muone}, dashed]
      {normalpdf(x,\muone,1)};
    % Setas e anotacoes.
    \path[o->, draw] (axis cs: -1,0.125)
      to[out=90, in=-90] (axis description cs: 0.2,0.5)
      node[Yellow, above] {this };
    \path[o->, draw] (axis cs: 1.25,0.25)
      to[out=45, in=-90] (axis description cs: 0.8,0.7)
      node[Blue, above] {that };
    \path[o->, draw] (axis cs: -2,0.025)
      to[out=0, in=180] (axis description cs: 0.7,0.4)
      node[Red, right] {who};
    \path[o->, draw] (axis cs: 1.8,0.025)
      to[out=0, in=180] (axis description cs: 0.7,0.4);
    \node[below right, text width=10cm]
      at (axis description cs: 0, -0.075) {you \\ and me};
  \end{axis}
\end{tikzpicture}


\section{回归投影}


\begin{tikzpicture}[
  >=stealth,
  % x={(0.866cm, -0.5cm)},
  % y={(0.866cm, 0.5cm)},
  % z={(0cm, 1cm)},
  % scale=1.0,
  % inner sep=0pt,
  % outer sep=2pt,
  axis/.style={->},
  vec/.style={thick, ->},
  every node/.style={color=black}]

  \coordinate (O) at (-0.75,0.25,-0.5);
  \draw[dotted, draw=darkgreen, fill=darkgreen!20]
    (-2,0,-3.5) -- (3,0,-3.5) -- (4,0,3) -- (-1,0,3) -- cycle;
  % \draw[axis] (O) -- +(3, 0, 0) node [right] {x};
  % \draw[axis] (O) -- +(0, 2, 0) node [right] {y};
  % \draw[axis] (O) -- +(0, 0, 2) node [above] {z};
  \draw[vec] (O) -- (2, 2, 0) node [above] {$y$};
  \draw[vec, darkgreen] (O) -- (2, 0, 0)
    node[right] {$\hat{y}=X\hat{\beta}$};
  \draw[vec, red] (O) -- (1.5, 0, 1.5)
    node[below] {$\bar{y}$};
  \draw[red, dotted] (1.5,0,1.5) -- (3, 0, 3)
    node[below] {$C(\mathbf{1})$};
  \draw[dashed, darkgreen, sloped] (2, 0, 0) -- (2,2,0)
    node[midway, below] {$||y-X\hat{\beta}||$};
  \draw[red, dashed, sloped] (2,2,0) -- (1.5, 0, 1.5)
    node[midway, above] {$||y-\bar{y}||$};
  \node[above left] at (3,0,-3.5) {$C(\mathbf{X})$};
\end{tikzpicture}


\section{3D}
\label{sec:3d}

\begin{tikzpicture}[
  x={(0.707cm,0.707cm)},
  z={(0cm,1cm)},
  y={(-0.866cm,0.5cm)}]

  \draw[->] (-2,0,0) -- (2,0,0) node[right] {x};
  \draw[->] (0,-2,0) -- (0,2,0) node[left] {y};
  \draw[->] (0,0,-2) -- (0,0,12) node[above] {z};
  \draw (1,0,0)
    \foreach \z in {0,0.1,...,10}{
      -- ({cos(\z*100)},{sin(\z*100)},{\z})
    };
  \node[rotate=90,right=1cm] at (0,0,12) {normal rotation};
\end{tikzpicture}

\begin{tikzpicture}[
  x={(0.707cm,0.707cm)},
  z={(0cm,1cm)},
  y={(-0.866cm,0.5cm)}]

  \draw[->] (-2,0,0) -- (2,0,0) node[right] {x};
  \draw[->] (0,-2,0) -- (0,2,0) node[left] {y};
  \draw[->] (0,0,-2) -- (0,0,12) node[above] {z};
  \draw (1,0,0)
    \foreach \z in {0,0.1,...,10}{
      -- ({cos(\z*200)},{sin(\z*100)},{\z})
    };
  \node[rotate=90,right=1cm] at (0,0,12) {WTF?};
\end{tikzpicture}

\begin{tikzpicture}[
  x={(0.707cm,0.707cm)},
  z={(0cm,1cm)},
  y={(-0.866cm,0.5cm)}]

  \draw[->] (-2,0,0) -- (2,0,0) node[right] {x};
  \draw[->] (0,-2,0) -- (0,2,0) node[left] {y};
  \draw[->] (0,0,-2) -- (0,0,12) node[above] {z};
  \draw (1,0,0)
    \foreach \z in {0,0.1,...,10}{
      -- ({cos(\z*189)},{sin(\z*91)},{\z})
    };
    \foreach \z in {0,0.1,...,9.9}{
      \fill[gray,opacity=0.2]
      ({cos(\z*189)},{sin(\z*91)},0) -- 
      ({cos(\z*189)},{sin(\z*91)},{\z}) --
      ({cos((\z+0.1)*189)},{sin((\z+0.1)*91)},{\z+0.1}) --
      ({cos((\z+0.1)*189)},{sin((\z+0.1)*91)},0) -- cycle;
    }
  \node[rotate=90,right=1cm] at (0,0,12) {to floor, gray};
\end{tikzpicture}

\begin{tikzpicture}[
  x={(0.707cm,0.707cm)},
  z={(0cm,1cm)},
  y={(-0.866cm,0.5cm)}]

  \draw[->] (-2,0,0) -- (2,0,0) node[right] {x};
  \draw[->] (0,-2,0) -- (0,2,0) node[left] {y};
  \draw[->] (0,0,-2) -- (0,0,12) node[above] {z};
  \foreach \z in {0,0.1,...,9.9}{
    \pgfmathtruncatemacro{\mycolorpercentage}{\z/0.099}
    \fill[red!\mycolorpercentage!blue,opacity=0.1]
      ({cos(\z*210)},{sin(\z*42)},0) --
      ({cos(\z*210)},{sin(\z*42)},{\z}) --
      ({cos((\z+0.1)*210)},{sin((\z+0.1)*42)},{\z+0.1}) --
      ({cos((\z+0.1)*210)},{sin((\z+0.1)*42)},0) -- cycle;
  }
  \node[rotate=90,right=1cm] at (0,0,12) {to floor, color gradient};
\end{tikzpicture}

\begin{tikzpicture}[
  x={(0.707cm,0.707cm)},
  z={(0cm,1cm)},
  y={(-0.866cm,0.5cm)}]

  \draw[->] (-2,0,0) -- (2,0,0) node[right] {x};
  \draw[->] (0,-2,0) -- (0,2,0) node[left] {y};
  \draw[->] (0,0,-2) -- (0,0,12) node[above] {z};
  \draw (1,0,0)
    \foreach \z in {0,0.1,...,10}{
      -- ({cos(\z*207)},{sin(\z*101)},{\z})
    };
  \foreach \z in {0,0.1,...,10}{
    \draw[opacity=0.5,gray]
      ({cos(\z*207)},{sin(\z*101)},{\z}) -- (0,0,\z);
  }
  \node[rotate=90,right=1cm] at (0,0,12) {radial lines};
\end{tikzpicture}

\begin{tikzpicture}[
  x={(0.707cm,0.707cm)},
  z={(0cm,1cm)},
  y={(-0.866cm,0.5cm)}]

  \draw[->] (-2,0,0) -- (2,0,0) node[right] {x};
  \draw[->] (0,-2,0) -- (0,2,0) node[left] {y};
  \draw[->] (0,0,-2) -- (0,0,12) node[above] {z};
  \draw (1,0,0)
    \foreach \z in {0,0.1,...,10}{
      -- ({cos(\z*237)},{sin(\z*111)},{\z})
    };
  \foreach \z in {0,0.1,...,9.9}{
    \fill[opacity=0.5,gray]
    (0,0,\z) --
    ({cos(\z*237)},{sin(\z*111)},{\z}) --
    ({cos((\z+0.1)*237)},{sin((\z+0.1)*111)},{(\z+0.1)}) --
    (0,0,{(\z+0.1)}) -- cycle;
  }
  \node[rotate=90,right=1cm] at (0,0,12) {radial surfaces, gray};
\end{tikzpicture}

\begin{tikzpicture}[
  x={(0.707cm,0.707cm)},
  z={(0cm,1cm)},
  y={(-0.866cm,0.5cm)}]

  \draw[->] (-2,0,0) -- (2,0,0) node[right] {x};
  \draw[->] (0,-2,0) -- (0,2,0) node[left] {y};
  \draw[->] (0,0,-2) -- (0,0,12) node[above] {z};
  \draw (1,0,0)
    \foreach \z in {0,0.1,...,10}{
      -- ({cos(\z*37)},{sin(\z*219)},{\z})
    };
  \foreach \z in {0,0.1,...,9.9}{
    \pgfmathtruncatemacro{\mycolorpercentage}{\z/0.099}
    \fill[opacity=0.3,orange!\mycolorpercentage!cyan]
      (0,0,\z) --
      ({cos(\z*37)},{sin(\z*219)},{\z}) --
      ({cos((\z+0.1)*37)},{sin((\z+0.1)*219)},{(\z+0.1)}) --
      (0,0,{(\z+0.1)}) -- cycle;
  }
  \node[rotate=90,right=1cm] at (0,0,12) {radial, color gradient};
\end{tikzpicture}




\end{document}