
\documentclass{standalone}
\usepackage[UTF8,punct]{ctex}

\usepackage{graphicx}
\usepackage{color}
\usepackage{xcolor}
\definecolor{darkgreen}{rgb}{0.13,0.53,0.53}
\definecolor{darkblue}{HTML}{224271}

\usepackage{amsfonts, amsmath} 

\usepackage{ifthen}
\usepackage{bm}

\usepackage{pgfplots}
\usepackage{pgfplotstable}
\usepackage{pgfcalendar}
\pgfplotsset{compat=newest}
\usepgfplotslibrary{groupplots}
\usepgfplotslibrary{fillbetween}

\usepackage{tikz}
\usetikzlibrary{
  arrows,
  automata,
  backgrounds,
  calc,
  calendar,
  chains,
  decorations,
  decorations.pathreplacing,
  decorations.pathmorphing,
  decorations.markings,
  decorations.text,
  er,
  external,
  fadings,
  fit,
  fixedpointarithmetic,
  fpu,
  folding,
  intersections,
  matrix,
  mindmap,
  lindenmayersystems,
  positioning,
  patterns,
  petri,
  plotmarks,
  shapes,
  shadows,
  shadings,
  snakes,
  spy,
  topaths,
  trees,
  turtle,
  through}

%%%%%%%%%%%%%%%%%%%%%

\begin{document}
%\begin{tikzpicture}

\newcommand{\mx}[1]{\mathbf{\bm{#1}}} % Matrix command
\newcommand{\vc}[1]{\mathbf{\bm{#1}}} % Vector command

% We need layers to draw the block diagram
\pgfdeclarelayer{background}
\pgfdeclarelayer{foreground}
\pgfsetlayers{background,main,foreground}

% Define a few styles and constants
\tikzstyle{sensor}=[draw, fill=blue!20, text width=5em, 
    text centered, minimum height=2.5em]
\tikzstyle{ann} = [above, text width=5em]
\tikzstyle{naveqs} = [sensor, text width=6em, fill=red!20, 
    minimum height=12em, rounded corners]
\def\blockdist{2.3}
\def\edgedist{2.5}

\begin{tikzpicture}
    \node (naveq) [naveqs] {Navigation equations};
    % Note the use of \path instead of \node at ... below. 
    \path (naveq.140)+(-\blockdist,0) node (gyros) [sensor] {Gyros};
    \path (naveq.-150)+(-\blockdist,0) node (accel) [sensor] {Accelero-meters};
    
   
    \path [draw, ->] (gyros) -- node [above] {$\vc{\omega}_{ib}^b$} 
        (naveq.west |- gyros) ;
    % We could simply have written (gyros) .. (naveq.140). However, it's
    % best to avoid hard coding coordinates
    \path [draw, ->] (accel) -- node [above] {$\vc{f}^b$} 
        (naveq.west |- accel);
    \node (IMU) [below of=accel] {IMU};
    \path (naveq.south west)+(-0.6,-0.4) node (INS) {INS};
    \draw [->] (naveq.50) -- node [ann] {Velocity } + (\edgedist,0) 
        node[right] {$\vc{v}^l$};
    \draw [->] (naveq.20) -- node [ann] {Attitude} + (\edgedist,0) 
        node[right] { $\mx{R}_l^b$};
    \draw [->] (naveq.-25) -- node [ann] {Horisontal position} + (\edgedist,0)
        node [right] {$\mx{R}_e^l$};
    \draw [->] (naveq.-50) -- node [ann] {Depth} + (\edgedist,0) 
        node[right] {$z$};
    
    % Now it's time to draw the colored IMU and INS rectangles.
    % To draw them behind the blocks we use pgf layers. This way we  
    % can use the above block coordinates to place the backgrounds   
    \begin{pgfonlayer}{background}
        % Compute a few helper coordinates
        \path (gyros.west |- naveq.north)+(-0.5,0.3) node (a) {};
        \path (INS.south -| naveq.east)+(+0.3,-0.2) node (b) {};
        \path[fill=yellow!20,rounded corners, draw=black!50, dashed]
            (a) rectangle (b);
        \path (gyros.north west)+(-0.2,0.2) node (a) {};
        \path (IMU.south -| gyros.east)+(+0.2,-0.2) node (b) {};
        \path[fill=blue!10,rounded corners, draw=black!50, dashed]
            (a) rectangle (b);
    \end{pgfonlayer}
\end{tikzpicture}




  
%\end{tikzpicture}
\end{document}